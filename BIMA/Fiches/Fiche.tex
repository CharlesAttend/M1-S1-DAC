\documentclass{article}
\usepackage[utf8]{inputenc}
\usepackage[a4paper, margin=2.5cm]{geometry}
\usepackage{graphicx}
\usepackage[french]{babel}

\usepackage[default,scale=0.95]{opensans}
\usepackage[T1]{fontenc}
\usepackage{amssymb} %math
\usepackage{amsmath}
\usepackage{amsthm}
\usepackage{systeme}

\usepackage{hyperref}
\hypersetup{
    colorlinks=true,
    linkcolor=blue,
    filecolor=magenta,      
    urlcolor=cyan,
    pdftitle={Fiche BIMA},
    % pdfpagemode=FullScreen,
    }
\urlstyle{same} %\href{url}{Text}

\theoremstyle{plain}% default
\newtheorem{thm}{Théorème}[section]
\newtheorem{lem}[thm]{Lemme}
\newtheorem{prop}[thm]{Proposition}
\newtheorem*{cor}{Corollaire}
%\newtheorem*{KL}{Klein’s Lemma}

\theoremstyle{definition}
\newtheorem{defn}{Définition}[section]
\newtheorem{exmp}{Exemple}[section]
% \newtheorem{xca}[exmp]{Exercise}

\theoremstyle{remark}
\newtheorem*{rem}{Remarque}
\newtheorem*{note}{Note}
%\newtheorem{case}{Case}



\title{Fiche BIMA}
\author{Charles Vin}
\date{Décembre 2022}

\begin{document}
\maketitle
\tableofcontents
Gausienne 2D : 
\[
    \frac{1}{\sqrt[]{2 \pi \sigma }} e^{- \frac{x^2 + y^2}{2 \sigma ^2}}
.\]

\section{Edge Detection with filtering}
\begin{itemize}
    \item Un bord dans une image peut ressembler à une marche d'escalier ou à une rampe : il est plus ou moins nette
    \item On regarde la direction du gradient : $ \left\| \nabla f \right\| = \sqrt[]{(\frac{\delta f}{\delta x})^2 + (\frac{\delta f}{\delta y})^2} $ que l'ont normalise $ \frac{\nabla f}{\left\| \nabla f \right\| } $ pour obtenir un vecteur unitaire
    \item Par une méthode mathématique obscure nommée différence finis, on peut approximer les dérivés des images pas une convolution 
\end{itemize}
\subsection{Sobel Edge filter}
\[
    G_x = \begin{bmatrix}
        1 & 0 & -1 \\
        2 & 0 & -2 \\
        1 & 0 & -1 
    \end{bmatrix} = \begin{bmatrix}
        1 \\
        2 \\
        1
    \end{bmatrix} \times \begin{bmatrix}
        1 & 0 & -1
    \end{bmatrix}
.\]

\[
    G_y = G_x^T
.\]


\begin{itemize}
    \item la réponse impulsionnel de Sobel est en faite composé d'une matrice qui approxime la gaussienne et la matrice de dérivation horizontale $ \big(\begin{smallmatrix}
        1 & 0 & -1
    \end{smallmatrix}\big) $ 
    \item $ \left\| G \right\| = \sqrt[]{G_x^2 + G_y^2} $ cette norme est plus forte au niveau des contours (car dérivé d'un escalier $ = + \infty  $ )
\end{itemize}


\subsection{Second order}

\[
    \begin{pmatrix}
        0 & 1 & 0 \\
        1 & -4 & 1 \\
        0 & 1 & 0 \\
    \end{pmatrix} \text{ ou } \begin{pmatrix}
        1 & 1 & 1 \\
        1 & -8 & 1 \\
        1 & 1 & 1 \\
    \end{pmatrix}
.\]
\begin{itemize}
    \item Ici on regarde quand la dérivée seconde s'annule pour trouver le max de la dérivé
    \item On utilise un laplacien pour approximer la matrice hessienne $ \Delta f = \frac{\partial ^2 f}{\partial x^2} + \frac{\partial ^2f}{\partial y^2} $ 
    \item Detecter les passages par zéros : \begin{itemize}
        \item Fenetre 3x3 $\rightarrow$ max et min
        \item zéro crossing = $max > 0, min < 0, max - min > S$
    \end{itemize}
    \item Plus précis et moins sensible à la threshold que gradient
    \item \textbf{Pas} invariant par rotation ! 
    \item Thick edge
    \item bruit ++ $\rightarrow$ filtrage nécessaire $\rightarrow$ \textbf{On peut combiner les deux en une convolution} avec le laplacien de la gaussienne 2D
\end{itemize}

\subsection{Approche pyramidale}
Filtre gaussien $\rightarrow$ subsample 2 $\rightarrow$ filtre $\rightarrow$...

\subsection{Canny-Deriche}
Filtre gaussien plus optimisé + implémentation récursive possible pour éviter de faire deux fois la convolution(x et y)

\subsection{Post processing}
\begin{itemize}
    \item Binarization Threshold : thick edge + bruit ou missed detection $\Rightarrow$ Gaussian smoothing 
    \item Gaussian smoothing + Threshold : \begin{itemize}
        \item flou ++ = moins de bruit // thick edge (imprecise localization)
        \item Flou -- = bruit // bonne localisation
    \end{itemize}
    \item Non maxima suppression \begin{itemize}
        \item Arrondie sur une des 8 directions 
        \item Interpolation à partir des deux voisins 
        \item $\rightarrow$ Bord fin 
    \end{itemize}
\end{itemize}


\section{Corner Detection}
\subsection{Moravec keypoint detection}



\end{document}