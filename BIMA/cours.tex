\documentclass{article}
\usepackage[utf8]{inputenc}
\usepackage[a4paper, margin=2.5cm]{geometry}
\usepackage{graphicx}
\usepackage[french]{babel}

\usepackage[default,scale=0.95]{opensans}
\usepackage[T1]{fontenc}
\usepackage{amssymb} %math
\usepackage{amsmath}
\usepackage{amsthm}
\usepackage{systeme}

\usepackage{hyperref}
\hypersetup{
    colorlinks=true,
    linkcolor=blue,
    filecolor=magenta,      
    urlcolor=cyan,
    pdftitle={BIMA},
    % pdfpagemode=FullScreen,
    }
\urlstyle{same} %\href{url}{Text}

\theoremstyle{plain}% default
\newtheorem{thm}{Théorème}[section]
\newtheorem{lem}[thm]{Lemme}
\newtheorem{prop}[thm]{Proposition}
\newtheorem*{cor}{Corollaire}
%\newtheorem*{KL}{Klein’s Lemma}

\theoremstyle{definition}
\newtheorem{defn}{Définition}[section]
\newtheorem{exmp}{Exemple}[section]
% \newtheorem{xca}[exmp]{Exercise}

\theoremstyle{remark}
\newtheorem*{rem}{Remarque}
\newtheorem*{note}{Note}
%\newtheorem{case}{Case}



\title{Base du traitement d'image}
\author{Charles Vin}
\date{S1-2022}

\begin{document}
\maketitle

\underline{Nouveau cours du 13/09} \\

https://www-master.ufr-info-p6.jussieu.fr/parcours/ima/bima/

TP : correction avec échantillonnage random
Une semaine pour les faire, il veut qu'on aille plus loins que les question ? 
40\% CC et 50\% Exam, pas de DS1 

Prise en compte de la perception dans le traitement d'image (sinon on ferait du traitement de signal) : exemple des illusions d'optiques

\subsection{Encodage d'image}
\subsubsection{RBG}
RGB = un cube \\
Limite de RGB :
\begin{itemize}
    \item Les 3 canaux sont très corrélé $\rightarrow$ Information redondante 
    \item Problème pour utiliser des distances euclidienne : certaine couleurs sont proche en distance mais sont pas du tout pareil
\end{itemize}
Extension : \begin{itemize}
    \item D'autre représentation : ACP 
    \item Utiliser des model plus proche de l'humais : HSV 
\end{itemize}

\subsubsection{HSV}
HSV = un cone 
\begin{itemize}
    \item Value : Moyenne de RGB 
    \item Hue : De la trigo comme c'est un cercle 
    \item Saturation : De la trigo comme c'est un cercle 
\end{itemize}

Brightness : juste la moyenne des valeur de l'image \\
Contrast : C'est lié à la distance entre le min et le max qu'on définie pour le niveau de gris. On regarde la distance entre les valeur min et max, on peut utiliser l'écart type. 

3 niveaux d'analyse/de compréhension des images:
\begin{itemize}
    \item Low : image => image \begin{itemize}
        \item Compression 
        \item Restauration (retirer le bruit)
        \item Filtrage (trouver uniquement les contours)
        \item Segmentation (un pixel = un label)
    \end{itemize}
    \item Mid : image => Attributes 
    \item High : image => understanding (semantic description)
\end{itemize}
Proche du processing fait par les réseaux de neurone

\underline{Nouveau cours du 20/09} \\

\section{Basic image transformations}
VOIR LE DIAPO AVANT 
\begin{itemize}
    \item Transformation affine
    \item Translation
    \item Change of scale
    \item Rotation
    \item Linear transformation
\end{itemize}

Cordonnée homogène : \begin{itemize}
    \item On ajoute une nouvelle coordonne qui vaut 1 partout. Ainsi maintenant une translation dans $ \mathbb{R}^2 $ peuvent s'exprimer comme une opération linéaire dans $ \mathbb{R}^3 $ 
    \item Ca permet d'être rapide comme les multiplications de matrice sont cablés dans les GPUs. 
    \item Problème : En pivotant une image on créé des trous dans l'image (un losange dans un carré) et les pixels ne sont plus carrés. Deux solutions : \begin{itemize}
        \item Direct transformation: pixels coordinates in the output image are determined from pixels coordinates in the input image \\
        $\rightarrow$ can generate missing data or superposition.
        \item Inverse transformation: pixels coordinates in the input image are determined from pixels coordinates in the output image \\
        $\rightarrow$ can generate superposition and also missing data (due to bounded spatial domain).
        \item IMAGE DIAPO 12
        \item Exemple DIAPO 13
    \end{itemize}
\end{itemize}

Méthode d'interpolation: 
Two examples of basic interpolation methods
\begin{itemize}
    \item Nearest neighbor: pixel value is given by the value of the nearest neighboring pixel
    \item Bilinear interpolation: pixel value is determined from the 4 nearest neighboring pixels using a bilinear interpolation (inconvéniant : lisse l'image, la rend flou)
\end{itemize}
Many other interpolation methods: B-splines, Hermite interpolation polynomials, ...
IMAGE DIAPO 14
\begin{note}[]
    Globalement deux opération en traitement d'image : Soit on moyenne, soit on intègre. On verra dans le TD3.
\end{note}

Application des transformations géométrique : 
\begin{itemize}
    \item Suivre les objets sur deux caméras différente (un passant qui passe d'une caméra à l'autre) Belle image dans le diapo 17. Tout peut se passer avec une matrice $ 3 \times 3 $.
    \item Faire le lien entre une carte dessiné et une image satellite. On peut détecter les points d’intérêt, ou en connaissant précisément la longitude et la latitude de chaque pixel
    \item Application médical : Superposition de plusieurs modalité provenant de plusieurs capteur. Même reconstruire en 3D.
    \item Compression vidéo : principe on essaye de prédire l'image suivant à partir de la précédente avec une fonction. Bref on regarde uniquement les pixel qui se déplace. 
\end{itemize}

\subsection{Operation between images}
Application : \begin{itemize}
    \item Soustration et réduction de bruit : Soit deux images $ I, I^n, I - I^n $ permet de détecter le bruit. On obtient alors que des pixels noir lorsqu'il n'y a pas de différence. Le reste est du bruit. \\
    $\rightarrow$ Permet également de trouver les changements dans le temps, détection de mouvement. (image diapo 31-32)
\end{itemize}

\subsection{Image thresholding}
\begin{itemize}
    \item Thresholding: reduction of image values to few levels of intensity
    \item Binarization: image values are reduced to two intensity levels
    \item Binary thresholding, defined by: 
    \[
        k' = \begin{cases}
        k_1 &\text{ si } k \leq S\\
        k_2 &\text{ si } k \geq S\\
        \end{cases} 
    .\]
    with $ k_1, k_2 $  and $ S $ (threshold) are levels of intensity
\end{itemize}
$\rightarrow$ Highlights regions but does not enhance the image

\subsection{Image enhancement}
\begin{defn}[]
    \begin{itemize}
        \item 
    \end{itemize}
\end{defn}
Trois catégories : \begin{itemize}
    \item Pixel level enhancement : \begin{itemize}
        \item the image brightness or contrast is modified
        \item no spatial information, only radiometric value of the visited pixel is considered
    \end{itemize}
    \item Local enhancement : Prend en compte le voisinage
    \item Enhancement in the frequency domain : on prend l'image, on quitte la représentation spacial et on bascule dans le domaine fréquentiel pour faire des changements avant de revenir dans domaine spacial.
\end{itemize}

Application : \begin{itemize}
    \item Regions to highlight
    \item Images that are too bright or too dark
    \item Intensity levels should be changed in order to make some details in the image more visible
    \item Modify image brightness
    \item Increase contrast (see Lecture 1)
    \item Pixel-level enhancement is closely related to histogram
    transformation
    \item IMAGE COOL DIAPO 41
\end{itemize}

\subsection{Histogram}
\begin{defn}[Histogram]
    Histogram is an array / function describing the image values (intensities / gray values / colors) distribution. 
    Provides image-specific information, such as:
    \begin{itemize}
        \item The statistical distribution of image values
        \item Minimal and maximal image values, moyenne, médiane
        \item \textbf{No spatial information at all (exemple diapo 44)}
    \end{itemize}
    Formule : 
    \[
        H(k) = \text{ complexe pour rien}
    .\]
    En réalité on parcours l'image et on regarde à quel interval il appartient et on incrémente celui-ci.
    
    \begin{exmp}[]
        Voir l'image diapo 43.
    \end{exmp}
\end{defn}
\begin{defn}[Normalized Histogram]
    Permet d'approximer la densité de la loi. Car l'histograme normale n'est pas conforme à la définition d'une densité (intégrale égal à 1). 

    Function Hn representing the probability (occurrence frequenc here) for a pixel to have a given value k 
    \[
        H_n(k) = \frac{H(k)}{N \times M}
    .\]
    with N and M are the image dimensions. Maintenant elle est à valeur dans $ \in  [0,1] $ 
\end{defn}
\begin{defn}[Cumulative Histogram]
    Permet d'avoir une estiamtion de la fonction de répartition. On fait la somme cumulative 
    \[
        H_C(k) = \sum_{i \leq k}^{H(i)}
    .\]
\end{defn}

\subsection{Image négative}
\begin{defn}[]
    negative of the image obtained by the negative transformation in the range of $[0, L - 1]$:
    \[
        k' = L - 1 -k
    .\]
    with $ l $  the dynamic range of the image (number of intensity levels)
\end{defn}

\subsection{Transformation}
\subsubsection{Logarithmic transformation}
\begin{defn}[]
    Low values increase, high values decrease: allows increasing
    the contrast in dark parts of the image.    
    \[
        k' = \log_{} (k)
    .\]
    Améliore le contraste dans les parties les plus sombre de l'image. EXEMPLE DIAPO 51
\end{defn}

\subsubsection{Exponential transformation}
\begin{defn}[]
    Low values decrease, high values increase: allows increasing the contrast in bright parts of the image. 
    \[
        k' = e^k
    .\]
\end{defn}

\subsection{Opération sur les histograme }
\begin{defn}[Histogram Translation]
    Changes the \textbf{brightness} of an image, leaving the contrast unchanged. The new image is brighter or darker. Useful for images having a low dynamic range.
    \[
        k' = k+t
    .\]
\end{defn}

\begin{defn}[Affine transformation]
    
\end{defn}

\begin{defn}[Image Normalization]
    On ramène les valeurs entre 0 et $ L-1 $.
    \begin{itemize}
        \item Let kmin and kmax be the minimal and maximal intensity levels
        of an image, respectively:
        \item Transformation: 
        \[
            k' = \frac{L - 1}{k_{max} - k_{min}} (k - k_{min})
        .\]
        \item After transformation, $ k' \in [0, L-1] $, contrast is maximal
        \item No loss of information (same number of intensity levels)
        \item Before visualization, an image is often normalized (but not necessarily)
    \end{itemize}
    SUPER EXEMPLE DANS LE DIAPO 59  
\end{defn}

\begin{defn}[Linear transformation with saturation]
    Cette fois ci on vas étirer l'histogramme uniquement dans un intervale donnée. Voir diapo 60 pour plus d'information.

    Cette fois il y a de la perte d'information.
\end{defn}

\begin{defn}[Histogram equalization]
    On vas le coder ahah. On prend l'histogramme et on le rend plat (équidistribué).\\ 
    \textbf{Perte d'information} mais bien pour la visualisation.
    \begin{itemize}
        \item Each intensity level is represented in the same proportion
        \item Regions of lower local contrast gain a higher contrast
        \item Global contrast increases
    \end{itemize}
    L'idée de la méthode : On fusions les pixels peu représenté avec les plus représenter.
    \[
        k' = Int(\frac{L-1}{N*M}H_c(k))
    .\]
    Avec \begin{itemize}
        \item $ L $ the image dynamic range
        \item $ N $ and $ M $  the image size
        \item $ H_c(k) $ the cumulative histogram
        \item $ Int $ rounding to the nearest integer
    \end{itemize}
    Super exemple dans le diapo 64. 
\end{defn}

\begin{note}[Histogram stretching versus histogram equalization: same operation?]
    \begin{itemize}
        \item Stretching: changes the bins distribution in the histogram, but not their size
        \item Equalization: changes the bins distribution in the histogram and their size
    \end{itemize}
\end{note}

\begin{note}[Application]
    \begin{itemize}
        \item Image mosaic : a target image and a base of small image.
        \item Face recognition : On suppose que les histogram d'image faciale ont la même loi de proba.
        \item Segmentation : En fusionnant beaucoup beaucoup les classes on peut segmenter en 5 couleurs (exemple diapo 70). Ou on utilise des algo de clustering
    \end{itemize}
\end{note}











\end{document}