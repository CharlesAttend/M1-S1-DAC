\documentclass{article}
\usepackage[utf8]{inputenc}
\usepackage[a4paper, margin=2.5cm]{geometry}
\usepackage{graphicx}
\usepackage[french]{babel}

\usepackage[default,scale=0.95]{opensans}
\usepackage[T1]{fontenc}
\usepackage{amssymb} %math
\usepackage{amsmath}
\usepackage{amsthm}
\usepackage{systeme}

\usepackage{hyperref}
\hypersetup{
    colorlinks=true,
    linkcolor=blue,
    filecolor=magenta,      
    urlcolor=cyan,
    pdftitle={BIMA},
    % pdfpagemode=FullScreen,
    }
\urlstyle{same} %\href{url}{Text}

\theoremstyle{plain}% default
\newtheorem{thm}{Théorème}[section]
\newtheorem{lem}[thm]{Lemme}
\newtheorem{prop}[thm]{Proposition}
\newtheorem*{cor}{Corollaire}
%\newtheorem*{KL}{Klein’s Lemma}

\theoremstyle{definition}
\newtheorem{defn}{Définition}[section]
\newtheorem{exmp}{Exemple}[section]
% \newtheorem{xca}[exmp]{Exercise}

\theoremstyle{remark}
\newtheorem*{rem}{Remarque}
\newtheorem*{note}{Note}
%\newtheorem{case}{Case}



\title{Base du traitement d'image}
\author{Charles Vin}
\date{S1-2022}

\begin{document}
\maketitle

\underline{Nouveau cours du 13/09} \\

https://www-master.ufr-info-p6.jussieu.fr/parcours/ima/bima/

TP : correction avec échantillonnage random
Une semaine pour les faire, il veut qu'on aille plus loins que les question ? 
40\% CC et 50\% Exam, pas de DS1 

Prise en compte de la perception dans le traitement d'image (sinon on ferait du traitement de signal) : exemple des illusions d'optiques

\subsection{Encodage d'image}
\subsubsection{RBG}
RGB = un cube \\
Limite de RGB :
\begin{itemize}
    \item Les 3 canaux sont très corrélé $\rightarrow$ Information redondante 
    \item Problème pour utiliser des distances euclidienne : certaine couleurs sont proche en distance mais sont pas du tout pareil
\end{itemize}
Extension : \begin{itemize}
    \item D'autre représentation : ACP 
    \item Utiliser des model plus proche de l'humais : HSV 
\end{itemize}

\subsubsection{HSV}
HSV = un cone 
\begin{itemize}
    \item Value : Moyenne de RGB 
    \item Hue : De la trigo comme c'est un cercle 
    \item Saturation : De la trigo comme c'est un cercle 
\end{itemize}

Brightness : juste la moyenne des valeur de l'image \\
Contrast : C'est lié à la distance entre le min et le max qu'on définie pour le niveau de gris. On regarde la distance entre les valeur min et max, on peut utiliser l'écart type. 

3 niveaux d'analyse/de compréhension des images:
\begin{itemize}
    \item Low : image => image \begin{itemize}
        \item Compression 
        \item Restauration (retirer le bruit)
        \item Filtrage (trouver uniquement les contours)
        \item Segmentation (un pixel = un label)
    \end{itemize}
    \item Mid : image => Attributes 
    \item High : image => understanding (semantic description)
\end{itemize}
Proche du processing fait par les réseaux de neurone

\underline{Nouveau cours du 20/09} \\

\section{Basic image transformations}
VOIR LE DIAPO AVANT 
\begin{itemize}
    \item Transformation affine
    \item Translation
    \item Change of scale
    \item Rotation
    \item Linear transformation
\end{itemize}

Cordonnée homogène : \begin{itemize}
    \item On ajoute une nouvelle coordonne qui vaut 1 partout. Ainsi maintenant une translation dans $ \mathbb{R}^2 $ peuvent s'exprimer comme une opération linéaire dans $ \mathbb{R}^3 $ 
    \item Ca permet d'être rapide comme les multiplications de matrice sont cablés dans les GPUs. 
    \item Problème : En pivotant une image on créé des trous dans l'image (un losange dans un carré) et les pixels ne sont plus carrés. Deux solutions : \begin{itemize}
        \item Direct transformation: pixels coordinates in the output image are determined from pixels coordinates in the input image \\
        $\rightarrow$ can generate missing data or superposition.
        \item Inverse transformation: pixels coordinates in the input image are determined from pixels coordinates in the output image \\
        $\rightarrow$ can generate superposition and also missing data (due to bounded spatial domain).
        \item IMAGE DIAPO 12
        \item Exemple DIAPO 13
    \end{itemize}
\end{itemize}

Méthode d'interpolation: 
Two examples of basic interpolation methods
\begin{itemize}
    \item Nearest neighbor: pixel value is given by the value of the nearest neighboring pixel
    \item Bilinear interpolation: pixel value is determined from the 4 nearest neighboring pixels using a bilinear interpolation (inconvéniant : lisse l'image, la rend flou)
\end{itemize}
Many other interpolation methods: B-splines, Hermite interpolation polynomials, ...
IMAGE DIAPO 14
\begin{note}[]
    Globalement deux opération en traitement d'image : Soit on moyenne, soit on intègre. On verra dans le TD3.
\end{note}

Application des transformations géométrique : 
\begin{itemize}
    \item Suivre les objets sur deux caméras différente (un passant qui passe d'une caméra à l'autre) Belle image dans le diapo 17. Tout peut se passer avec une matrice $ 3 \times 3 $.
    \item Faire le lien entre une carte dessiné et une image satellite. On peut détecter les points d’intérêt, ou en connaissant précisément la longitude et la latitude de chaque pixel
    \item Application médical : Superposition de plusieurs modalité provenant de plusieurs capteur. Même reconstruire en 3D.
    \item Compression vidéo : principe on essaye de prédire l'image suivant à partir de la précédente avec une fonction. Bref on regarde uniquement les pixel qui se déplace. 
\end{itemize}

\subsection{Operation between images}
Application : \begin{itemize}
    \item Soustration et réduction de bruit : Soit deux images $ I, I^n, I - I^n $ permet de détecter le bruit. On obtient alors que des pixels noir lorsqu'il n'y a pas de différence. Le reste est du bruit. \\
    $\rightarrow$ Permet également de trouver les changements dans le temps, détection de mouvement. (image diapo 31-32)
\end{itemize}

\subsection{Image thresholding}
\begin{itemize}
    \item Thresholding: reduction of image values to few levels of intensity
    \item Binarization: image values are reduced to two intensity levels
    \item Binary thresholding, defined by: 
    \[
        k' = \begin{cases}
        k_1 &\text{ si } k \leq S\\
        k_2 &\text{ si } k \geq S\\
        \end{cases} 
    .\]
    with $ k_1, k_2 $  and $ S $ (threshold) are levels of intensity
\end{itemize}
$\rightarrow$ Highlights regions but does not enhance the image

\subsection{Image enhancement}
\begin{defn}[]
    \begin{itemize}
        \item 
    \end{itemize}
\end{defn}
Trois catégories : \begin{itemize}
    \item Pixel level enhancement : \begin{itemize}
        \item the image brightness or contrast is modified
        \item no spatial information, only radiometric value of the visited pixel is considered
    \end{itemize}
    \item Local enhancement : Prend en compte le voisinage
    \item Enhancement in the frequency domain : on prend l'image, on quitte la représentation spacial et on bascule dans le domaine fréquentiel pour faire des changements avant de revenir dans domaine spacial.
\end{itemize}

Application : \begin{itemize}
    \item Regions to highlight
    \item Images that are too bright or too dark
    \item Intensity levels should be changed in order to make some details in the image more visible
    \item Modify image brightness
    \item Increase contrast (see Lecture 1)
    \item Pixel-level enhancement is closely related to histogram
    transformation
    \item IMAGE COOL DIAPO 41
\end{itemize}

\subsection{Histogram}
\begin{defn}[Histogram]
    Histogram is an array / function describing the image values (intensities / gray values / colors) distribution. 
    Provides image-specific information, such as:
    \begin{itemize}
        \item The statistical distribution of image values
        \item Minimal and maximal image values, moyenne, médiane
        \item \textbf{No spatial information at all (exemple diapo 44)}
    \end{itemize}
    Formule : 
    \[
        H(k) = \text{ complexe pour rien}
    .\]
    En réalité on parcours l'image et on regarde à quel interval il appartient et on incrémente celui-ci.
    
    \begin{exmp}[]
        Voir l'image diapo 43.
    \end{exmp}
\end{defn}
\begin{defn}[Normalized Histogram]
    Permet d'approximer la densité de la loi. Car l'histograme normale n'est pas conforme à la définition d'une densité (intégrale égal à 1). 

    Function Hn representing the probability (occurrence frequenc here) for a pixel to have a given value k 
    \[
        H_n(k) = \frac{H(k)}{N \times M}
    .\]
    with N and M are the image dimensions. Maintenant elle est à valeur dans $ \in  [0,1] $ 
\end{defn}
\begin{defn}[Cumulative Histogram]
    Permet d'avoir une estiamtion de la fonction de répartition. On fait la somme cumulative 
    \[
        H_C(k) = \sum_{i \leq k}^{H(i)}
    .\]
\end{defn}

\subsection{Image négative}
\begin{defn}[]
    negative of the image obtained by the negative transformation in the range of $[0, L - 1]$:
    \[
        k' = L - 1 -k
    .\]
    with $ l $  the dynamic range of the image (number of intensity levels)
\end{defn}

\subsection{Transformation}
\subsubsection{Logarithmic transformation}
\begin{defn}[]
    Low values increase, high values decrease: allows increasing
    the contrast in dark parts of the image.    
    \[
        k' = \log_{} (k)
    .\]
    Améliore le contraste dans les parties les plus sombre de l'image. EXEMPLE DIAPO 51
\end{defn}

\subsubsection{Exponential transformation}
\begin{defn}[]
    Low values decrease, high values increase: allows increasing the contrast in bright parts of the image. 
    \[
        k' = e^k
    .\]
\end{defn}

\subsection{Opération sur les histograme }
\begin{defn}[Histogram Translation]
    Changes the \textbf{brightness} of an image, leaving the contrast unchanged. The new image is brighter or darker. Useful for images having a low dynamic range.
    \[
        k' = k+t
    .\]
\end{defn}

\begin{defn}[Affine transformation]
    
\end{defn}

\begin{defn}[Image Normalization]
    On ramène les valeurs entre 0 et $ L-1 $.
    \begin{itemize}
        \item Let kmin and kmax be the minimal and maximal intensity levels
        of an image, respectively:
        \item Transformation: 
        \[
            k' = \frac{L - 1}{k_{max} - k_{min}} (k - k_{min})
        .\]
        \item After transformation, $ k' \in [0, L-1] $, contrast is maximal
        \item No loss of information (same number of intensity levels)
        \item Before visualization, an image is often normalized (but not necessarily)
    \end{itemize}
    SUPER EXEMPLE DANS LE DIAPO 59  
\end{defn}

\begin{defn}[Linear transformation with saturation]
    Cette fois ci on vas étirer l'histogramme uniquement dans un intervale donnée. Voir diapo 60 pour plus d'information.

    Cette fois il y a de la perte d'information.
\end{defn}

\begin{defn}[Histogram equalization]
    On vas le coder ahah. On prend l'histogramme et on le rend plat (équidistribué).\\ 
    \textbf{Perte d'information} mais bien pour la visualisation.
    \begin{itemize}
        \item Each intensity level is represented in the same proportion
        \item Regions of lower local contrast gain a higher contrast
        \item Global contrast increases
    \end{itemize}
    L'idée de la méthode : On fusions les pixels peu représenté avec les plus représenter.
    \[
        k' = Int(\frac{L-1}{N*M}H_c(k))
    .\]
    Avec \begin{itemize}
        \item $ L $ the image dynamic range
        \item $ N $ and $ M $  the image size
        \item $ H_c(k) $ the cumulative histogram
        \item $ Int $ rounding to the nearest integer
    \end{itemize}
    Super exemple dans le diapo 64. 
\end{defn}

\begin{note}[Histogram stretching versus histogram equalization: same operation?]
    \begin{itemize}
        \item Stretching: changes the bins distribution in the histogram, but not their size
        \item Equalization: changes the bins distribution in the histogram and their size
    \end{itemize}
\end{note}

\begin{note}[Application]
    \begin{itemize}
        \item Image mosaic : a target image and a base of small image.
        \item Face recognition : On suppose que les histogram d'image faciale ont la même loi de proba.
        \item Segmentation : En fusionnant beaucoup beaucoup les classes on peut segmenter en 5 couleurs (exemple diapo 70). Ou on utilise des algo de clustering
    \end{itemize}
\end{note}

\underline{Nouveau cours du 27/09} \\
15 min de retard, before diapo 11. 

En gros la transformé de Fourier permet de représenter les fonctions périodique avec des coordonnes dans l'espaces infinis des fonctions périodiques. Pourquoi \begin{itemize}
    \item Pour écrire les signaux de manière plus compact
    \item Pour pouvoir les compresser
    \item Pour pouvoir les comparer ensuite. 
    \item CCL : DIAPO 14 
\end{itemize}

Transformé de Fourier : La seule différence c'est qu'on obtient un continuum de fréquence 

\paragraph*{Interprétation du signal} : On obtient une représentation de notre fréquence en terme de fréquence (haute ou basse)\begin{itemize}
    \item On utilise le module $ \left| X(f) \right|  $ pour obtenir la quantité de la fréquence pure $ f $ présente dans le signal $ x $.
    \item La phase : L'angle de $ X(f) $ vue dans ce cas comme un vecteur du plans complexe
\end{itemize}
Le signal $ x $ peut être reconstruit à partir de sa transformé de Fourier $ X $. 

\begin{defn}[Convolution]
    Une sorte de moyenne locale de $ g $ pondéré par les valeur de $ f $. Opération commutative, distributive, associative.
\end{defn}

\paragraph*{Retour fourier diapo 26} 
Liste des propriétés :\begin{itemize}
    \item Linéaire
    \item Time scaling : Quand on multiplie un signal par un scalaire, on le stretch et il s'étend. A l'inverse sa transformé de Fourier se rétrécis 
    \item Time shifting : Si on translate le temps, on ne change pas la transformé
    \item Frequency shifting : translater les fréquence, translate la transformé de Fourier
    \item Théorème de la convolution : \begin{itemize}
        \item  Convolution assez complexe à calculer $ O(n^2) $ 
        \item Avec Fourier la convolution est super plus simple à calculer $ O(n \log_{}n) $ 
        \item Voir formule dans le diapo 26 si nécessaire
    \end{itemize}
\end{itemize}
\begin{exmp}[]
    Exemple d'une transformé de Fourier avec une fonction porte.
\end{exmp}
\begin{exmp}[]
    Exemple d'une transformé de Fourier avec une Gaussienne $\rightarrow$ On retrouve une gaussienne ! Damn c'est fou
\end{exmp}
\begin{exmp}[]
    Exemple d'une transformé de Fourier avec une fonction de Dirac $ \delta (t) $. C'est l'élément neutre de la convolution. Au final, on pondère la moyenne mobile sur un unique point $\rightarrow$ ca donne la valeur de la fonction en ce point. 
\end{exmp}
\begin{exmp}[]
    Exemple d'une transformé de Fourier de $ \sin  $ et $ \cos  $. C'est relativement facile en utilisant la formule d'Euler, qui donne deux exponentiel complexe, et une exponentielle complexe c'est une fonction de Dirac avec Fourier.
\end{exmp}

\paragraph*{Fourier en 2D}
Globalement la même chose sauf qu'on intègre pour chaque dimension. On obtient en couple de fréquence, et pour le plot on utilise l'intensité lumineuse pour représenter l'amplitude.

\textbf{Inportant:}la transformé de Fourier donne des information sur l'orientation des objects, leur taille, ect EXEMPLE DIAPO 53

DIAPO 54, quelque fonction très utile. Et diapo 55 illustration de ce que fait la fonction $ fftshift() $

Diapo 56 : On passe au log pour améliorer la visualisation. 

Quelques Applications : \begin{itemize}
    \item Débruiter : en supprimant les hautes fréquences
    \item Compression : Les petits détails sont haute fréquences donc on peut les supprimer.
    \item Obtenir les directions des choses ect
    \item Filtrer : edge detection, point d'intérêt, ... 
\end{itemize}

Désavantage d'une représentation fréquentielle : \begin{itemize}
    \item A cause de l'invariance par translation, on a pas d'information spacial.
    \item Signaux non dérivable : Avec que des portes (=des contours nettes dans images), pour l'approcher avec des $ \sin  $ il faut une infinité de $ \sin  $ 
\end{itemize}

\underline{Nouveau cours du 04/10} \\
\section{2chantillonage}
On doit échantilloné les signaux continues \begin{itemize}
    \item Le fenétrage : on limite le support (domaine du signal) dans le temps
    \item 
\end{itemize}

\subsection{Fenêtrage}
Classiquement on multiplie pas une fonction porte.

\subsubsection{1D}
La multiplication par une fonction rectangle $ \Leftrightarrow $ faire la convolution dans le domaine fréquentiel.

Mais visiblement ça complique pas mal la transformé en Fourier (voir image diapo 6 ). On retrouve des sinus cardianux. On verra en TD comment faire et si la taille de la fenêtre a une conséquence.

\subsubsection{2D}
Ici on multiplie par deux fonctions portes, ...
Mais on a la même chose en 3D au final, on retrouve le sinus cardinal 

Quand on prend une fenetre petite, la Fourier s'élargie fortement, ce qui peut la rendre illisible. Avec une fenetre large on tend vers une diract (diapo 12)

\subsection{Echantillonage}
On prend une mesure tous les x temps. Si fréquence trop faible, on n'arrive pas à reconstuire l'aspect continue, mais si trop grand on prend beaucoup de ressource de calcul.

\subsubsection{1D}
Pour faire ça on multiplie notre signal par un peigne de Dirac $ \mathrm{III} $, c'est une collection de Dirac. 

Qu'est ce qui se passe dans le domaine fréquentiel ? La TF d'un peigne de Dirac c'est une peigne de Dirac, c'est plutôt contre intuitif. Mais en gros ça rend la TF périodique. 

\paragraph*{Perte d'information}
Y'a-t-il une perte d'information pendant l'échantillonnage ? Oui et non \begin{itemize}
    \item Non Si on prend un signal borné séquentiellement. C'est à dire que la transformé de Fourier a un support borné sur l'axe des abscises
    \item Sinon oui il y a perte
\end{itemize}
Liste des signaux band-limité \begin{itemize}
    \item Rect non
    \item sinc oui
    \item sin oui 
    \item cos oui 
    \item Dirac oui 
    \item BIEN ÉCRIRE DIAPO
\end{itemize}

\paragraph*{Quel fréquence d'échantillonnage ?} Pour pouvoir échantillons sans perte un signal, il faut utilise une fréquence max deux fois plus grande que la fréquence max du signal. C'est le théorème de Shannon. DIAPO 23 J'suis pas sur de ce que j'ai écrit. 

\paragraph*{Reconstuire un signal échantillonné ?} On doit re-fenetrer. PAs compris vraiment mais en gros à la fin faux pas obtenir de courbe qui se chevauche. 

\paragraph*{CCL} : Si Shannon est vérifié, toute l'information est dans la TF, on peut reconstruire le signal avec la formule de reconstruction de Shannon

\paragraph*{Aliasing} Si on ne vérifie pas les dégradation on a un phénomène d'aliasing. Difficile à prédire. 
\begin{exmp}[Cas d'une fonction sin]
    DIAPO 30 TO 39
    Ici ça illustre les deux cas (correct et incorrect).\begin{itemize}
        \item Avec $ 4f_0 $, c'est suffisant pour reconstruire le signal.
        \item Diapo 33 : à droite on a plot la formule d'interpolation de Shannon. 
        \item On peut prendre une fréquence d'échantillonnage plus élevé mais ça change rien.
        \item DIAPO 35 : fréquence trop basse $\rightarrow$ mauvais fenêtrage 
        \item Diapo 37 : on peut voir une divergence des points rouge d'échantillonnage. 
    \end{itemize}
\end{exmp}

\subsubsection{2D}
Pareil, toujours les condition de Shannon : bande limité, fréquence d'échantillonnage. Plus restrictif car dans les deux direction. Le peigne de Dirac est représenté par une grille d'échantillonnage.

On peut utiliser une grille cartésienne ou hexagonale (qui a certain avantage). Diapo 45

Transformé de Fourier : il dit c'est pareil. On vas répéter le spectre dans toute les direction. Théorème de Shannon : pareil dans les deux directions. 

Reconstruction : toujours possible avec une plus grosse formule.

\paragraph*{Signaux à bande limité : Réalistique ? } En théorie non, jsp pk, mais en pratique on peut ignorer les fréquences de bruit créée. \begin{itemize}
    \item En pratique ça marche bien avec les signaux lisse et stationnaire comme un ECG.
    \item Pour les images naturels, chaque contours représente une forte fréquence (non lisse).
    \item Exemple d'aliasing diapo 58 59 60
\end{itemize}

\paragraph*{Anti-aliasing}
\begin{itemize}
    \item Filtrer toute les fréquences qu'on peut pas représenter avec un filtre passe bas à partir de la moitié de la fréquence d'échantillonnage. 
    \item Utile lorsque qu'on downsample une image. En retirant certain pixel, on perds certaine fréquence donc on crée de l'aliasing (?). Du coup on applique également un filtre aliasing.
\end{itemize}

\subsection{Quantification}
Après l'échantillonnage on applique la quantification. But : réduire le nombre de bits nécessaire pour encoder l'image (je crois). Deux type de quantifier \begin{itemize}
    \item Scalaire : chaque échantillon est quantifié
    \item Vecteur : une séquence d'échantillon est quantifié
\end{itemize}

\subsubsection{Scalar quantification}
Aucune idée de quoi on parle.

\section{Transformé de Fourier discrète}
C'est l'échantillonnage de la transformé de Fourier du signal échantillonné.  \begin{enumerate}
    \item Transformé de Fourier du signal échantillonné. Calcul DIAPO 70 me semble pas mal.
    \item On windows
    \item On resample la DFT pour obtenir N valeur (comme donné à l'origine.) : ça resemble beaucoup à la série de fourier mais avec un $ k $ discret plutôt qu'un $ x $ continue. 
\end{enumerate}
Si on a un signal de $ N $ sample, alors la FT nous rend $ N $ valeur tel que : la formule diapo 75. 

Il vas vite mais globalement pas de différence en 2D 

\section{What you need to know}
The difference between:
\begin{enumerate}
    \item Fourier transform of a continuous signal $\rightarrow$ a continuous function
    \item Fourier transform of a windowed signal $\rightarrow$ a continuous function
    \item Fourier Transform of a discrete (sampled) signal $\rightarrow$ again a continuous function
    \item discrete Fourier transform of a discrete signal $\rightarrow$ a discrete function
\end{enumerate}


\end{document}