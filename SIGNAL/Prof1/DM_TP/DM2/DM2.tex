\documentclass{article}
\usepackage[utf8]{inputenc}
\usepackage[a4paper, margin=2.5cm]{geometry}
\usepackage{graphicx}
\usepackage[french]{babel}

\usepackage[default,scale=0.95]{opensans}
\usepackage[T1]{fontenc}
\usepackage{amssymb} %math
\usepackage{amsmath}
\usepackage{amsthm}
\usepackage{systeme}
\usepackage{xcolor}

\usepackage{hyperref}
\hypersetup{
    colorlinks=true,
    linkcolor=blue,
    filecolor=magenta,      
    urlcolor=cyan,
    pdftitle={Compte Rendu TM2},
    % pdfpagemode=FullScreen,
    }
\urlstyle{same} %\href{url}{Text}

\theoremstyle{plain}% default
\newtheorem{thm}{Théorème}[section]
\newtheorem{lem}[thm]{Lemme}
\newtheorem{prop}[thm]{Proposition}
\newtheorem*{cor}{Corollaire}
%\newtheorem*{KL}{Klein’s Lemma}

\theoremstyle{definition}
\newtheorem{defn}{Définition}[section]
\newtheorem{exmp}{Exemple}[section]
% \newtheorem{xca}[exmp]{Exercise}

\theoremstyle{remark}
\newtheorem*{rem}{Remarque}
\newtheorem*{note}{Note}
%\newtheorem{case}{Case}



\title{Compte Rendu TM2}
\author{Charles Vin}
\date{30/09}

\begin{document}
\maketitle

\section{Exercice 1}
\begin{enumerate}
    \item Racine du polynôme : $ \{-1, -3\} $ 
    \begin{align*}
        X(S) &= \frac{10 (S+1)}{S^2 + 4S + 3} \\
            &= \frac{10 (S+1)}{(S+1)(S+3)} \\
            &= \frac{10}{S+3}\\
    \end{align*}
    D'après la ligne cinq du tableau on a 
    \[
        x(t) = e^\lambda u(t) \Leftrightarrow X(S) = \frac{1}{S - \lambda }
    .\]
    Dans notre cas $ \lambda = -3 $. Conclusion : 
    \begin{align*}
        X(S) &= \frac{10 (S+1)}{S^2 + 4S + 3} \\
        \downarrow \mathcal{L}^{-1}& \\
            x(t) &= 10e^{-3t}u(t)
    \end{align*} 
    
    \item \begin{align*}
        X(S) &= \frac{10 (S+1)e^{-2S}}{S^2 + 4S + 3} \\
        &= \frac{10 (S+1)e^{-2S}}{(S+1)(S+3)} \\
        &= \frac{10e^{-2S}}{(S+3)} 
    \end{align*}
    Nous somme dans le même cas que précédemment avec un time shifting de $ 2 $. Conclusion :
    \begin{align*}
        X(S) &= \frac{10 (S+1)e^{-2S}}{S^2 + 4S + 3}\\
        \downarrow \mathcal{L}^{-1}& \\
            x(t) &= 10e^{-3(t-2)}u(t-2)
    \end{align*}
    
    \item Les racines de $ S(S^2 + 10S + 16) $ sont $ \{0, -2, -8\} $ 
    \begin{align*}
        X(S) &= \frac{20}{S(S^2 + 10S + 16)} \\
            &= \frac{20}{S(S+2)(S+8)} \\
            &= \frac{A}{S} + \frac{B}{S+2} + \frac{C}{S+8} \\
    \end{align*}
    Trouvons $ A, B, C $ : 
    \begin{align*}
        &\frac{20}{S(S+2)(S+8)} = \frac{A}{S} + \frac{B}{S+2} + \frac{C}{S+8} \\
        \Leftrightarrow & 20 = A(S+2)(S+8) + B(S+8)S + C(S+2)S \\ 
        \Leftrightarrow & 20 = AS^2 + 10AS + 16A + BS^2 + 8BS + CS^2 + 2CS \\
        \Leftrightarrow & 20 = (A + B + C)S^2 + (10A + 8B + 2C)S + 16A
    \end{align*}
    Par identification : 
    \begin{align*}
        &\begin{cases}
            A + B + C = 0 \\
            10A + 8B + 2C = 0 \\
            16A = 20
        \end{cases} \\
        \Leftrightarrow &
        \begin{cases}
            B = -5/4 - C \\
            8(-5/4 - C) + 2C = -50/4 \\
            A = 20/16 = 5/4
        \end{cases} \\
        \Leftrightarrow &
        \begin{cases}
            B = -5/4 - C \\
            -6C = -50/4 + 40/4 = -10/4 \\
            A = 5/4 \\
        \end{cases} \\
        \Leftrightarrow &
        \begin{cases}
            B = -5/3 \\
            C = 5/12 \\
            A = 5/4 \\
        \end{cases}
    \end{align*}
    \begin{align*}
        X(S) &= \frac{20}{S(S^2 + 10S + 16)}\\
        \downarrow \mathcal{L}^{-1}& \\
        x(t) &= \frac{5}{4}u(t) - \frac{5}{3}e^{-2t}u(t) + \frac{5}{12}e^{-8t}u(t) \\
            &= u(t)(\frac{5}{4} - \frac{5}{3}e^{-2t} + \frac{5}{12}e^{-8t})
    \end{align*}
\end{enumerate}

\section{Exercice 2}
\begin{align*}
    H(S) &= \frac{20S(S+100)}{(S+2)(S+10)} \\
        &= \frac{20*100}{2*10} \frac{S(1 + S/100)}{(1+S/2)(1+S/10)} \\
        &= 100 * \frac{S(1 + S/100)}{(1+S/2)(1+S/10)} \\
\end{align*}
Calcul des niveaux de base
\begin{itemize}
    \item Le gain : $ 20 \log_{} 100  = 40db$ 
    \item Le phase : $ = 0 $ car $ 100 > 0 $  
\end{itemize}
Grâce à ces calculs préalable, on peut maintenant tracer le diagramme de Bode du gain dans la figure \ref{BodeGain}. Voici sa légende : \begin{itemize}
    \item \color{black} Niveau de base : 40db \color{orange}
    \item Zero à l'origine: $ (S) \rightarrow 20 \frac{db}{dec}$ \color{red}
    \item $ (1 + \frac{S}{100}) \rightarrow +20 \frac{db}{dec}$ \color{green}
    \item $ (1 + \frac{S}{2}) \rightarrow -20 \frac{db}{dec}$ \color{blue}
    \item $ (1 + \frac{S}{10}) \rightarrow -20 \frac{db}{dec}$ \color{yellow}
    \item Résultante \color{black}
\end{itemize}

\begin{figure}[htbp]
    \centering
    \includegraphics*[width=\textwidth]{./gain.png}
    \caption{Diagramme de Bode du gain}
    \label{BodeGain}
\end{figure}

Puis on peut tracer le diagramme de la phase sur la figure \ref{BodePhase}. Voici sa légende : 
\begin{itemize}
    \item \color{black} Niveau de base : 0 \color{orange}
    \item Zero à l'origine: $ (S) \rightarrow +\frac{\pi}{2}$ \color{red}
    \item $ (1 + \frac{S}{100}) \rightarrow +\frac{\pi}{4}$ sur $ [10,1000] $ \color{green}
    \item $ (1 + \frac{S}{2}) \rightarrow -\frac{\pi}{4}$ sur $ [0.2,20] $ \color{blue}
    \item $ (1 + \frac{S}{10}) \rightarrow -\frac{\pi}{4}$ sur $ [1,100] $ \color{yellow}
    \item Résultante \color{black}
\end{itemize}

\begin{figure}[htbp]
    \centering
    \includegraphics*[width=\textwidth]{./phase.png}
    \caption{Diagramme de Bode de la phase}
    \label{BodePhase}
\end{figure}
\subsection{MatLab}
Voici deux diagramme de Bode:
\begin{itemize}
    \item Le premier dans la figure \ref{BodeMatLab} fut généré avec la fonction $ bode() $ de la toolbox de signal processing.
    \item Le deuxième généré avec le code indiqué.
\end{itemize}
\begin{figure}[htbp]
    \centering
    \includegraphics*[]{untitled.png}
    \caption{Figure généré avec la fonction $ bode() $ }
    \label{BodeMatLab}
\end{figure}
\end{document}