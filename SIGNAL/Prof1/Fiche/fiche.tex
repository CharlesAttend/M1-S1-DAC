\documentclass{article}
\usepackage[utf8]{inputenc}
\usepackage[a4paper, margin=2.5cm]{geometry}
\usepackage{graphicx}
\usepackage[french]{babel}

\usepackage[default,scale=0.95]{opensans}
\usepackage[T1]{fontenc}
\usepackage{amssymb} %math
\usepackage{amsmath}
\usepackage{amsthm}
\usepackage{systeme}

\usepackage{hyperref}
\hypersetup{
    colorlinks=true,
    linkcolor=blue,
    filecolor=magenta,      
    urlcolor=cyan,
    pdftitle={Fiche SIGNAL},
    % pdfpagemode=FullScreen,
    }
\urlstyle{same} %\href{url}{Text}

\theoremstyle{plain}% default
\newtheorem{thm}{Théorème}[section]
\newtheorem{lem}[thm]{Lemme}
\newtheorem{prop}[thm]{Proposition}
\newtheorem*{cor}{Corollaire}
%\newtheorem*{KL}{Klein’s Lemma}

\theoremstyle{definition}
\newtheorem{defn}{Définition}[section]
\newtheorem{exmp}{Exemple}[section]
% \newtheorem{xca}[exmp]{Exercise}

\theoremstyle{remark}
\newtheorem*{rem}{Remarque}
\newtheorem*{note}{Note}
%\newtheorem{case}{Case}



\title{Fiche SIGNAL}
\author{Charles Vin}
\date{S1-2022}

\begin{document}
\maketitle
Exponentiel complexe : 
\[
    a + jb \to e^{(a+jb)t} = e^{at}e^{jbt} = e^{at}(\cos bt + j \sin bt)
.\]

\section{Composante paire et impaire d'une fonction}
Soit $ x(t) $ une fonction alors on peut écrire $ x(t) $ sous la forme suivante : 
\[
    x(t) = \frac{1}{2}(x(t) + x(-t)) + \frac{1}{2}(x(t) - x(-t))
.\]
Composante paire 
\[
    x_i(t) = \frac{1}{2}(x(t) - x(-t))
.\]
Composante impaire 
\[
    x_p(t) = \frac{1}{2}(x(t) + x(-t))
.\]

\section{Convolution}
\begin{align*}
    (f \star g)(t) &= \int _{-\infty }^{\infty }f(\tau )g(t-\tau )d\tau \\
                    &= \sum_{m=-\infty }^{+\infty} f[m]g[n - m]
\end{align*}

\begin{itemize}
    \item Inverser $ x(\tau ) $ en $ x(-\tau) $ par symétrie
    \item Décaler $ x(-\tau) $ de $ t $ donnant alors $ x(- \tau + t ) = x(t - \tau )$ 
    \item Différencier les différents cas possibles en croisant les graphs, \textbf{attention aux échelles de tes dessins} !!! 
    \item Pour chaque cas : calculer la convolution en utilisant les fonctions en jeux sur les bonnes bornes
\end{itemize}

\section{Système Linéaire, homogène et invariant}
\begin{itemize}
    \item Linéaire = Additivité + homogène 
    \item Additivité : \begin{itemize}
        \item Soit $ x_1(t) \to h(t) $ et $ x_2(t) \to h(t) $ $\rightarrow$ somme des deux sorties $ = y_1(t) $
        \item Soit $ x_1(t)+x_2(t) \to h(t) \to y_2(t) $
        \item Additif si $ y_1(t) = y_2(t) $  
    \end{itemize} 
    \item Homogène \begin{itemize}
        \item Soit $ x(t)*K \to h(t) \to y_1(t)$ 
        \item Soit $ x(t) \to h(t) \to *K \to y_2(t)$
        \item Homogène si $ y_1(t) = y_2(t) $
    \end{itemize}
    \item Invariance dans le temps \begin{itemize}
        \item Soit $ x(t) \to (t-T) \to h(t) \to y_1(t) $ 
        \item Soit $ x(t) \to h(t) \to t-T \to y_2(t) $ 
        \item Invariant si $ y_1(t) = y_2(t) $
    \end{itemize}
\end{itemize}

\section{Laplace, pôle, zéro, stabilité}
\begin{itemize}
    \item \textbf{Convolution} : $ \mathcal{L}\{x_1(t) \star x_2(t)\} = X_1(S) X_2(S)$ 
    \item Time shifting : $ \mathcal{L}\{x(t-t_0) u(t - t_0)\} = X(S)e^{-t_0 S}$ 
    \item Intégrale : $ \mathcal{L}\{\int_{0}^{t}x(t)dt\} = \frac{1}{S}X(S) $ 
\end{itemize}
\[
    H(S) = \frac{Y(S)}{X(S)}
.\]

Factoriser les polynômes au dénominateur et au numérateur. Puis pour appliquer Laplace inverse, on vérifie dans le tableau si on trouve une forme adapté. \\
Sinon on décompose en éléments simple pour obtenir une forme du type $ \frac{A}{S+c} \to _{\mathcal{L}^{-1}} A e^{-ct}u(t)$ \textbf{A vérifier dans le tableau}.\\
Décomposition en élément simple : 3 méthodes \begin{itemize}
    \item Identification : on remet tout sur le même dénominateur puis identification puis système d'équation
    \item Multiplier par un des facteurs : pratique si tous les dénominateurs ont la même puissance. On multiplie par un des dénominateur $\rightarrow$ annulation ou plusieurs des dénominateurs $\rightarrow$ remplacer $ x $ par le truc qui l'annule $ x=c $ $\rightarrow$ simplifier et hop on trouve le coef tout de suite
    \item Multiplier par $ x $  et faire la limite en $ +\infty  $  : puis appliquer le théorème du plus haut degré
\end{itemize}
\textbf{A voir si dans les annales il demande des décompositions en éléments simple du futur mais c'est là au cas où}

\subsection{Stabilité}
Soit $ H(S) = \frac{1}{(S+z_1)(S+z_2)\dots} $ une forme factorisé $ \to_{\mathcal{L}^{-1}} h(S)=\dots$. On définis la stabilité par en regardant si $ h(S) $ converge. $ \Leftrightarrow $ \begin{itemize}
    \item Stable : Si tous les pôles sont dans la partie gauche du plans complexe
    \item Instable : \begin{itemize}
        \item Au moins un pôle dans la partie droite du plan complexe
        \item Il existe au moins un pôle multiple sur l'axe imaginaire
    \end{itemize}
    \item Conditionnellement stable : Il existe un pôle simple sur l'axe imaginaire ($a=0$)
    \item Comme d'hab dessin dans OneNote
\end{itemize}


\section{Diagramme de Bode}
\begin{itemize}
    \item Module d'un complexe : $ \left| z \right| = \left| a + ib \right| = \sqrt[]{a^2 + b^2} $ 
    \item Argument : $ \arg z $ 
    \item Gain = $ 20 \log_{10} \left| H(Jw) \right| $
    \item Relire OneNote (SIGNAL/Prof1/TP2 toute fin) pour les détails du pourquoi 
\end{itemize}
La méthode pour une fonction $ H(s) $ 
\begin{enumerate}
    \item Mettre la fonction en forme normale 
    \begin{align*}
        H(S) &= K \frac{(S + a_1)(S+a_2)}{S(S+b_1)(S^2 + b_2 S + b_3)} \\
            &= \frac{K a_1 a_2}{b_1 b_3} * \frac{(1 + S/a_1)(1+S/a_2)}{S(1+S/b_1)(1 + S*b_2/b_2 + S^2/b_3)} \\ 
            &= c * \frac{1 + \frac{S}{c_1} + \dots}{(1 + \frac{S}{c_2})(1 + \frac{S}{c_3})\dots}
    \end{align*}
    \item Constante de départ \begin{itemize}
        \item Gain : $ 20 \log_{10} c $ 
        \item Phase : $ \begin{cases}
        0 &\text{si } c>0 \\
        \pi  &\text{si }c < 0 \\
        \end{cases}  $ 
    \end{itemize} 
    \item Les différents type de courbe à assembler sur le graph de Bode : $ 1 + \frac{S}{c_i} $ dans le figure \ref{bode}.
    \begin{figure}[!htbp]
        \centering
        \includegraphics*[width=.5\textwidth]{./bode.png}
        \caption{Les courbes de base pour Bode}
        \label{bode}
    \end{figure}
\end{enumerate}

\subsection{Diagramme de Bode inverse }
Bien observer et faire l'inverse je suppose ahah, s'entraîner pour voir \\
Dans le cours il a parlé de plein de truc, mais je sais pas si c'est nécessaire

\section{Filtre Butterworts}


\section{Filtre FIR : fenêtrage}




\end{document}