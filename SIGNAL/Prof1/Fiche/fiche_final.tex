\documentclass[9pt]{article}
\usepackage[utf8]{inputenc}
\usepackage[a4paper, margin=1cm]{geometry}
\usepackage{graphicx}
\usepackage[french]{babel}

% \usepackage[default,scale=0.30]{opensans}
\usepackage[T1]{fontenc}
\usepackage{amssymb} %math                                                                                                                                                  w
\usepackage{amsmath}
\usepackage{amsthm}
\usepackage{systeme}
\usepackage{bbm}
\usepackage{hyperref}
\usepackage{multicol}
\usepackage{sectsty}
\usepackage{titlesec}


% \titleformat{\section}[runin]{\footnotesize\bfseries\sffamily}{\thesection}{1em}{}

% \titleformat{\subsection}[runin]{\footnotesize\bfseries\sffamily}{\thesubsection}{1em}{}[]
% \titleformat{\subsubsection}[runin]{\footnotesize\bfseries\sffamily}{\thesubsubsection}{1em}{}[]
% \titleformat{\paragraph}[runin]{\footnotesize\bfseries\sffamily}{\theparagraph}{1em}{}[]
% \titleformat{\subparagraph}[runin]{\footnotesize\bfseries\sffamily}{\thesubparagraph}{1em}{}[]

\titleformat{\section}[runin]{\normalfont\footnotesize\bfseries}{\thesection}{1em}{}
\titleformat{\subsection}[runin]{\normalfont\footnotesize\bfseries}{\thesubsection}{1em}{}
\titleformat{\subsubsection}[runin]{\normalfont\footnotesize\bfseries}{\thesubsubsection}{1em}{}
\titleformat{\paragraph}[runin]{\normalfont\footnotesize\bfseries}{\theparagraph}{1em}{}
\titleformat{\subparagraph}[runin]{\normalfont\footnotesize\bfseries}{\thesubparagraph}{1em}{}

\titlespacing*{\section}{0pt}{0.8mm}{0.1mm}
\titlespacing*{\subsection}{0pt}{0.1mm}{0.1mm}

\begin{document}

\footnotesize   
\begin{multicols}{2}
\textbf{Manque : table type de filtre FIR, image bode, image gain FIR IIR, cercle trigo ?}
\subsection{Formule complexe}
\begin{align*}
    a + jb \to e^{(a+jb)t} &= e^{at}e^{jbt} = e^{at}(\cos bt + j \sin bt) \\
    z = a + jb \to \left| z \right| * e^{i \arg z} \\
    \tan(\arg z)&={\frac {\Im (z)}{\Re (z)}}={\frac {z-{\bar {z}}}{\mathrm {i} \left(z+{\bar {z}}\right)}} \\
    \arg z_1z_2 &= \arg z_1 + \arg z_2 \\
    \arg z_1^n &= n\arg z_1 \\
    \arg(az) &\equiv {\begin{cases}\arg z&{\text{si }}a>0\\(\arg z)+\pi &{\text{si }}a<0{\text{ ;}}\end{cases}}
\end{align*}

\subsection{Convolution}
\begin{align*}
    (f \star g)(t) &= \int _{-\infty }^{\infty }f(\tau )g(t-\tau )d\tau \\
                    &= \sum_{m=-\infty }^{+\infty} f[m]g[n - m]
\end{align*}

\subsection{Transformé de Laplace}
\begin{align*}
    X(S) &= \mathcal{L}\{x(t)\} \\
        &= \int_{-\infty }^{\infty }x(t)e^{-St}dt
\end{align*}

\begin{align*}
    X[Z] \to H[Z] \to &Y[Z] = X[Z]H[Z] \\
    \Leftrightarrow & H[Z] = \frac{Y[Z]}{X[Z]}
\end{align*}

\begin{itemize}
    \item \textbf{Convolution} : $ \mathcal{L}\{x_1(t) \star x_2(t)\} = X_1(S) X_2(S)$ 
    \item Time shifting : $ \mathcal{L}\{x(t-t_0) u(t - t_0)\} = X(S)e^{-t_0 S}$ 
    \item Intégrale : $ \mathcal{L}\{\int_{0}^{t}x(t)dt\} = \frac{1}{S}X(S) $ 
\end{itemize}
Factoriser les polynômes au dénominateur et au numérateur. Puis pour appliquer Laplace inverse, on vérifie dans le tableau si on trouve une forme adapté. \\
Sinon on décompose en éléments simple pour obtenir une forme du type $ \frac{A}{S+c} \to _{\mathcal{L}^{-1}} A e^{-ct}u(t)$.

Décomposition en élément simple $ \frac{A(X)}{B(X)} $ : \begin{enumerate}
    \item $ \deg A < \deg B $ sinon division euclidienne
    \item Si polynôme du second degrés sans racine réel au dénominateur, \textit{normalement} on le laisse et à la place d'une constante en haut on met une fonction affine. 
    \item 3 méthodes \begin{itemize}
        \item Identification : on remet tout sur le même dénominateur puis identification puis système d'équation
        \item Multiplier par un des facteurs : pratique si tous les dénominateurs ont la même puissance. On multiplie par un des dénominateur $\rightarrow$ annulation ou plusieurs des dénominateurs $\rightarrow$ remplacer $ x $ par le truc qui l'annule $ x=c $ $\rightarrow$ simplifier et hop on trouve le coef tout de suite
        \item Multiplier par $ x $  et faire la limite en $ +\infty  $  : puis appliquer le théorème du plus haut degré
    \end{itemize}
\end{enumerate}

\subsection{Composante paire et impaire d'une fonction}
Soit $ x(t) $ une fonction alors on peut écrire $ x(t) $ sous la forme suivante : 
\[
    x(t) = \frac{1}{2}(x(t) + x(-t)) + \frac{1}{2}(x(t) - x(-t))
.\]
Composante paire 
\[
    x_i(t) = \frac{1}{2}(x(t) - x(-t))
.\]
Composante impaire 
\[
    x_p(t) = \frac{1}{2}(x(t) + x(-t))
.\]
\subsection{Réalisation}
Utiliser une forme factorisé de $ H(S) $. \begin{itemize}
    \item Intégrateur/retard : $ 1/S $ 
    \item $ \oplus $ : additions
    \item $ \rhd $ : gains/multi
\end{itemize}

\paragraph*{Diagramme de Bode}
\begin{itemize}
    \item Gain = $ 20 \log_{10} \left| H(Jw) \right| $
    \item Relire OneNote (SIGNAL/Prof1/TP2 toute fin) pour les détails du pourquoi 
\end{itemize}
La méthode pour une fonction $ H(s) $ 
\begin{enumerate}
    \item Mettre la fonction en forme normale 
    \begin{align*}
        H(S) &= K \frac{(S + a_1)(S+a_2)}{S(S+b_1)(S^2 + b_2 S + b_3)} \\
            &= \frac{K a_1 a_2}{b_1 b_3} * \frac{(1 + S/a_1)(1+S/a_2)}{S(1+S/b_1)(1 + S*b_2/b_2 + S^2/b_3)} \\ 
            &= c * \frac{1 + \frac{S}{c_1} + \dots}{(1 + \frac{S}{c_2})(1 + \frac{S}{c_3})\dots}
    \end{align*}
    \item Constante de départ \begin{itemize}
        \item Gain : $ 20 \log_{10} c $ 
        \item Phase : $ \begin{cases}
        0 &\text{si } c>0 \\
        \pi  &\text{si }c < 0 \\
        \end{cases}  $ 
    \end{itemize} 

\end{enumerate}

\subsubsection{Stabilité continue}
On regarde uniquement les pôles pour la stabilité 
Plot partie réel $ \mathcal{R} $ et partie imaginaire $ Im $ 
\begin{itemize}
    \item Stable : Si tous les pôles sont dans la partie gauche du plans complexe
    \item Instable : \begin{itemize}
        \item Au moins un pôle dans la partie droite du plan complexe
        \item Il existe au moins un pôle multiple sur l'axe imaginaire
    \end{itemize}
    \item Conditionnellement stable : Il existe un pôle simple sur l'axe imaginaire ($a=0$)
\end{itemize}

\subsection{Stabilité discret}
\begin{itemize}
    \item Un système T.D. est stable si $\rightarrow$ touts les pôles sont à l'intérieur du cercle unitaire.
    \item Un système T.D. est instable si \begin{itemize}
        \item Au moins un pôle est à l'extérieur du cercle unitaire
        \item Il existe des pôles multiples sur le cercle unitaire
    \end{itemize}
    \item Un système T.D. est conditionnellement stable si $\rightarrow$ il existe des pôles simples sur le cercle unitaire.
\end{itemize}
\subsection{Type de filtre}
On peut éviter un diagramme de Bode en faisant : 
\begin{enumerate}
    \item Remplacer $ S=j \omega  $ 
    \item Placer les pôles dans le plans complexe et 
    \item $ \left| H(j \omega ) \right| = \frac{\prod_{}^{}\text{produit de la distance aux zéros}}{\prod_{}^{}\text{produit de la distance aux pôles}}  $ on observe la distance entre $ \omega $ et les pôles et zéros
    \item Classiquement : \begin{itemize}
        \item Vers $ \omega = \infty  $ on a $ \left| H(j \omega ) \right| = 1 $ 
        \item Avec $ \omega =  $ un zéros, il annule la fraction donc $ = 0 $ 
        \item En $ \omega = 0 $ on a le module des pôles et des zéros
    \end{itemize}
\end{enumerate}

\subsection{Filtre IIR : transformation bilinéaire}
On a $ \Omega _p, \Omega _s, R_p, A_s $ 
\begin{enumerate}
    \item Ordre du filtre : calculer toutes les constantes + les pôles + la fonction de transfert 
        \begin{itemize}
            \item Constante :
                \begin{align*}
                    w_p &= \frac{2}{T} \tan \frac{\Omega_p }{2} \\
                    w_s &= \frac{2}{T} \tan \frac{\Omega_s }{2} \\
                    N &= \left\lceil \frac{\log_{10} [(10^{R_p/10} - 1)/(10^{A_s/10} - 1)]}{2 \log_{10} \frac{w_p}{w_s}}\right\rceil \\
                    w_{cp} &= \frac{w_p}{(10^{R_p/10} - 1)^{1/2N}} \\
                    w_{cs} &= \frac{w_s}{(10^{A_s/10} - 1)^{1/2N}} \\
                    w_c &= \frac{w_{cp} + w_{cs}}{}
                \end{align*}
            \item Les pôles de $ H(S) $ 
                \begin{itemize}
                    \item $ N $ est impaire $\rightarrow$ On a l'argument de nos pôles $ \theta = \frac{k \pi }{N} \forall k \in \{0, \dots, 2N-1\} $ 
                    \item Si $ N $ paire $ \theta = \frac{(2k+1)\pi }{2N} $ 
                    \item $ p_i = w_c e^{j \theta_i } = w_c (\cos (\theta _i) + j \sin \theta _i)$ à calculer ! 
                    \item Tracer un plan avec le cercle de taille $ w_c $  et ne garder que les pôles stable à droite de l'ordonnée 
                \end{itemize}
            \item Écrire la fonction de transfert $ \frac{w_c^N}{(S-p_1)(S-p_2)\dots} $ avec les valeurs des $ p_i \in \mathbb{C}$ 
        \end{itemize}
    \item Position des pôles dans le plan $ Z $ : Conversion des pôles 
    \[
        S = \frac{2}{T} \frac{1 - Z^{-1}}{1 + Z^{-1}} \Leftrightarrow Z = \frac{ST + 2}{2 - ST}
    .\]
    On trouve des nouveaux pôles $ z_i = \frac{p_i*T + 2}{2 - p_i *T} $. Les tracer dans le plans complexe avec le cercle unitaire $\rightarrow$ montrer la stabilité.
    \item Écrire $ H(S) $ en replaçant $ S $ par $ S = \frac{2}{T} \frac{1 - Z^{-1}}{1 + Z^{-1}} $ puis simplifier le dénominateur à fond pour obtenir un polynôme. Vérifier que les racine de ce polynôme sont bien les pôles $ z_i $ calculé précédemment.
    \item Tracer approximativement la réponses en fréquence : voir figure \ref{rep_freq_iir_fir}
    % \begin{figure}[htbp]
    %     \centering
    %     \includegraphics*[width=.5\textwidth]{rep_freq_FIR_IIR.png}
    %     \caption{Réponse en fréquence filtre FIR et IIR}
    %     \label{rep_freq_iir_fir}
    % \end{figure}
    \item Proposer une réalisation matériel pour $ H(Z) $ : 
        \begin{itemize}
            \item Écrire $ H(Z) $ calculé à la question 3 sous une forme factorisé $ b_0 \frac{(1+ a_0 Z^{-1})\dots}{(1 + a_1 Z^{-1})\dots} $ On déclare des constantes. 
            \item Le numérateur = la sortie ; le dénominateur = l'entrée
            \item Faire le circuit : penser à développer les constantes au numérateur pour ne pas oublier de multiplieur
        \end{itemize}
\end{enumerate}

\subsection{Filtre FIR : Fenêtrage}
\begin{enumerate}
    \item Choisir la fenêtre : toujours rectangulaire je pense lol
    \item Ordre du filtre $ \Omega _s - \Omega _p = \frac{1.8 \pi }{M} $ puis partie supérieur. $ M - 1 =  $ ordre du filtre
    \item Coef du filtre : \begin{itemize}
        \item On vas fenêtrer sur l'intervale $ [-\frac{M-1}{2} ; \frac{M-1}{2}] $ en le divisant en $ M $ point.
        \item Coef : $ \Omega _c = \Omega _s - \Omega _p $ fréquence de coupure 
        \[
            h_i = \begin{cases}
                1s &\text{ si } x = 0\\
                \frac{\sin x \Omega _c}{x*\pi } &\text{ sinon}\\
                \end{cases}  
        .\]
        Utiliser la symétrie pour ne calculer que la moitié des points !
        \item Conclure en écrivant $ H(Z) $ et en factorisant les coefs symétrique
    \end{itemize}
    \item Réalisation matériel : On obtient un $ H(Z) = h_0(1+Z^{-3}) + h_1 (Z^{-1} + Z^{-2})$ Construire ce qu'il y a entre parenthèse puis multiplier par les facteurs avant de sommer le tout.
    \item Comparaison : FIR plus coûteux mais évite les problèmes de stabilité des pôles, n'a pas de boucle et est plus performant.
\end{enumerate}
\subsubsection{Type de filtre FIR}
\begin{itemize}
    \item \textbf{Type 1:} On peut réaliser tous les types de filtres passe-bas, passe-bande, passe-haut
    \item \textbf{Type 2:} Il existe un zéro à $ \Omega = \pi  $ $\rightarrow$ On ne peut pas réaliser un filtre passe-haut
    \item \textbf{Type 3:} Il existe un zéro à $ \Omega = 0 $ et à $ \Omega = \pi  $ $\rightarrow$ One ne peut pas réaliser un filtre passe-bas, ni un filtre passe-haut.
    \item \textbf{Type 4:} Il existe un zéro à $ \Omega = 0 $ $\rightarrow$ on ne peut pas réaliser un filtre passe-bas
\end{itemize}
% \begin{table}[!ht]
%     \centering
%     \begin{tabular}{|l|l|l|}
%     \hline
%         Symétrie de $h[n]$ & Nombre de coefficients $M$ & Type \\ \hline
%         Symétrie positive $ h_n = - h_{M-1-n} $ & Impaire & 1 \\ \hline
%         ~ & Paire & 2 \\ \hline
%         Symétrie négative : $ h_n = - h_{M-1-n} $  & Impaire & 3 \\ \hline
%         ~ & Paire & 4 \\ \hline
%     \end{tabular}
% \end{table}
\end{multicols}
\end{document}