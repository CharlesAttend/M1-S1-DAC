\documentclass{article}
\usepackage[utf8]{inputenc}
\usepackage[a4paper, margin=2.5cm]{geometry}
\usepackage{graphicx}
\usepackage[french]{babel}

\usepackage[default,scale=0.95]{opensans}
\usepackage[T1]{fontenc}
\usepackage{amssymb} %math
\usepackage{amsmath}
\usepackage{amsthm}
\usepackage{systeme}

\usepackage{hyperref}
\hypersetup{
    colorlinks=true,
    linkcolor=blue,
    filecolor=magenta,      
    urlcolor=cyan,
    pdftitle={Fiche SIGNAL},
    % pdfpagemode=FullScreen,
    }
\urlstyle{same} %\href{url}{Text}

\theoremstyle{plain}% default
\newtheorem{thm}{Théorème}[section]
\newtheorem{lem}[thm]{Lemme}
\newtheorem{prop}[thm]{Proposition}
\newtheorem*{cor}{Corollaire}
%\newtheorem*{KL}{Klein’s Lemma}

\theoremstyle{definition}
\newtheorem{defn}{Définition}[section]
\newtheorem{exmp}{Exemple}[section]
% \newtheorem{xca}[exmp]{Exercise}

\theoremstyle{remark}
\newtheorem*{rem}{Remarque}
\newtheorem*{note}{Note}
%\newtheorem{case}{Case}



\title{Fiche SIGNAL}
\author{Charles Vin}
\date{S1-2022}

\begin{document}
\maketitle
\tableofcontents


. \\
. \\
Formule complexe : 
\begin{align*}
    a + jb \to e^{(a+jb)t} &= e^{at}e^{jbt} = e^{at}(\cos bt + j \sin bt) \\
    z = a + jb \to \left| z \right| * e^{i \arg z} \\
    \tan(\arg z)&={\frac {\Im (z)}{\Re (z)}}={\frac {z-{\bar {z}}}{\mathrm {i} \left(z+{\bar {z}}\right)}} \\
    \arg z_1z_2 &= \arg z_1 + \arg z_2 \\
    \arg z_1^n &= n\arg z_1 \\
    \arg(az) &\equiv {\begin{cases}\arg z&{\text{si }}a>0\\(\arg z)+\pi &{\text{si }}a<0{\text{ ;}}\end{cases}}
\end{align*}

\begin{figure}[!htbp]
    \centering
    \includegraphics*[width=0.35\textwidth]{cercle_trigo.png}
\end{figure}
    
\section{Composante paire et impaire d'une fonction}
Soit $ x(t) $ une fonction alors on peut écrire $ x(t) $ sous la forme suivante : 
\[
    x(t) = \frac{1}{2}(x(t) + x(-t)) + \frac{1}{2}(x(t) - x(-t))
.\]
Composante paire 
\[
    x_i(t) = \frac{1}{2}(x(t) - x(-t))
.\]
Composante impaire 
\[
    x_p(t) = \frac{1}{2}(x(t) + x(-t))
.\]

\section{Convolution}
\begin{align*}
    (f \star g)(t) &= \int _{-\infty }^{\infty }f(\tau )g(t-\tau )d\tau \\
                    &= \sum_{m=-\infty }^{+\infty} f[m]g[n - m]
\end{align*}

\begin{itemize}
    \item Inverser $ x(\tau ) $ en $ x(-\tau) $ par symétrie
    \item Décaler $ x(-\tau) $ de $ t $ donnant alors $ x(- \tau + t ) = x(t - \tau )$ 
    \item Différencier les différents cas possibles en croisant les graphs, \textbf{attention aux échelles de tes dessins} !!! 
    \item Pour chaque cas : calculer la convolution en utilisant les fonctions en jeux sur les bonnes bornes
\end{itemize}

\section{Système Linéaire, homogène et invariant}
\begin{itemize}
    \item Linéaire = Additivité + homogène 
    \item Additivité : \begin{itemize}
        \item Soit $ x_1(t) \to h(t) $ et $ x_2(t) \to h(t) $ $\rightarrow$ somme des deux sorties $ = y_1(t) $
        \item Soit $ x_1(t)+x_2(t) \to h(t) \to y_2(t) $
        \item Additif si $ y_1(t) = y_2(t) $  
    \end{itemize} 
    \item Homogène \begin{itemize}
        \item Soit $ x(t)*K \to h(t) \to y_1(t)$ 
        \item Soit $ x(t) \to h(t) \to *K \to y_2(t)$
        \item Homogène si $ y_1(t) = y_2(t) $
    \end{itemize}
    \item Invariance dans le temps \begin{itemize}
        \item Soit $ x(t) \to (t-T) \to h(t) \to y_1(t) $ 
        \item Soit $ x(t) \to h(t) \to t-T \to y_2(t) $ 
        \item Invariant si $ y_1(t) = y_2(t) $
    \end{itemize}
\end{itemize}

\section{Signaux continue}
\subsection{Transformé de Laplace, pôle, zéro, stabilité}
On trace les pôles et les zéros dans le plans complexe ! 

Transformé de Laplace
\begin{align*}
    X(S) &= \mathcal{L}\{x(t)\} \\
        &= \int_{-\infty }^{\infty }x(t)e^{-St}dt
\end{align*}

\begin{align*}
    X[Z] \to H[Z] \to &Y[Z] = X[Z]H[Z] \\
    \Leftrightarrow & H[Z] = \frac{Y[Z]}{X[Z]}
\end{align*}

\begin{itemize}
    \item \textbf{Convolution} : $ \mathcal{L}\{x_1(t) \star x_2(t)\} = X_1(S) X_2(S)$ 
    \item Time shifting : $ \mathcal{L}\{x(t-t_0) u(t - t_0)\} = X(S)e^{-t_0 S}$ 
    \item Intégrale : $ \mathcal{L}\{\int_{0}^{t}x(t)dt\} = \frac{1}{S}X(S) $ 
\end{itemize}
Factoriser les polynômes au dénominateur et au numérateur. Puis pour appliquer Laplace inverse, on vérifie dans le tableau si on trouve une forme adapté. \\
Sinon on décompose en éléments simple pour obtenir une forme du type $ \frac{A}{S+c} \to _{\mathcal{L}^{-1}} A e^{-ct}u(t)$.

Décomposition en élément simple $ \frac{A(X)}{B(X)} $ : \begin{enumerate}
    \item $ \deg A < \deg B $ sinon division euclidienne
    \item Si polynôme du second degrés sans racine réel au dénominateur, \textit{normalement} on le laisse et à la place d'une constante en haut on met une fonction affine. 
    \item 3 méthodes \begin{itemize}
        \item Identification : on remet tout sur le même dénominateur puis identification puis système d'équation
        \item Multiplier par un des facteurs : pratique si tous les dénominateurs ont la même puissance. On multiplie par un des dénominateur $\rightarrow$ annulation ou plusieurs des dénominateurs $\rightarrow$ remplacer $ x $ par le truc qui l'annule $ x=c $ $\rightarrow$ simplifier et hop on trouve le coef tout de suite
        \item Multiplier par $ x $  et faire la limite en $ +\infty  $  : puis appliquer le théorème du plus haut degré
    \end{itemize}
\end{enumerate}

\subsubsection{Stabilité}
Soit $ H(S) = \frac{1}{(S+p_1)(S+p_2)\dots} $ une forme factorisé $ \to_{\mathcal{L}^{-1}} h(S)=\dots$. On définis la stabilité par en regardant si $ h(S) $ converge. $ \Leftrightarrow $ \\
On regarde uniquement les pôles pour la stabilité 
\begin{itemize}
    \item Stable : Si tous les pôles sont dans la partie gauche du plans complexe
    \item Instable : \begin{itemize}
        \item Au moins un pôle dans la partie droite du plan complexe
        \item Il existe au moins un pôle multiple sur l'axe imaginaire
    \end{itemize}
    \item Conditionnellement stable : Il existe un pôle simple sur l'axe imaginaire ($a=0$)
\end{itemize}


\subsection{Réponse en fréquence}
Soit $ H(S) $ une fonction de transfert, on remplace $ S = -j \omega $ puis phase $ \left| H(S) \right|  $ et gain $ \lfloor H(j \omega ) $ 

\paragraph*{Diagramme de Bode}
\begin{itemize}
    \item Gain = $ 20 \log_{10} \left| H(Jw) \right| $
    \item Relire OneNote (SIGNAL/Prof1/TP2 toute fin) pour les détails du pourquoi 
\end{itemize}
La méthode pour une fonction $ H(s) $ 
\begin{enumerate}
    \item Mettre la fonction en forme normale 
    \begin{align*}
        H(S) &= K \frac{(S + a_1)(S+a_2)}{S(S+b_1)(S^2 + b_2 S + b_3)} \\
            &= \frac{K a_1 a_2}{b_1 b_3} * \frac{(1 + S/a_1)(1+S/a_2)}{S(1+S/b_1)(1 + S*b_2/b_2 + S^2/b_3)} \\ 
            &= c * \frac{1 + \frac{S}{c_1} + \dots}{(1 + \frac{S}{c_2})(1 + \frac{S}{c_3})\dots}
    \end{align*}
    \item Constante de départ \begin{itemize}
        \item Gain : $ 20 \log_{10} c $ 
        \item Phase : $ \begin{cases}
        0 &\text{si } c>0 \\
        \pi  &\text{si }c < 0 \\
        \end{cases}  $ 
    \end{itemize} 
    \item Les différents type de courbe à assembler sur le graph de Bode : $ 1 + \frac{S}{c_i} $ dans le figure \ref{bode}.
    \begin{figure}[!htbp]
        \centering
        \includegraphics*[width=.5\textwidth]{./bode.png}
        \caption{Les courbes de base pour Bode}
        \label{bode}
    \end{figure}
\end{enumerate}

\paragraph*{Diagramme de Bode inverse}
Bien observer et faire l'inverse je suppose ahah, s'entraîner pour voir $\rightarrow$ EZ\\
Dans le cours il a parlé de plein de truc, mais je sais pas si c'est nécessaire

\subsection{Type de filtre}
Pour determiner le type de filtre d'une fonction de transfert $ H(S) $ , on regarde son gain $ \left| H(S) \right| $.

On peut éviter un diagramme de Bode en faisant : 
\begin{enumerate}
    \item Remplacer $ S=j \omega  $ 
    \item Placer les pôles dans le plans complexe et 
    \item $ \left| H(j \omega ) \right| = \frac{\prod_{}^{}\text{produit de la distance aux zéros}}{\prod_{}^{}\text{produit de la distance aux pôles}}  $ on observe la distance entre $ \omega $ et les pôles et zéros
    \item Classiquement : \begin{itemize}
        \item Vers $ \omega = \infty  $ on a $ \left| H(j \omega ) \right| = 1 $ 
        \item Avec $ \omega =  $ un zéros, il annule la fraction donc $ = 0 $ 
        \item En $ \omega = 0 $ on a le module des pôles et des zéros
    \end{itemize}
\end{enumerate}

\textbf{NON FAIRE UN BODE VITE FAIT si pas de $ \mathbb{C} $ } $ \uparrow  $ NE FONCTIONNE PAS DE OUF 

\subsection{Réalisation}
Utiliser une forme factorisé de $ H(S) $. \begin{itemize}
    \item Intégrateur/retard : $ 1/S $ 
    \item $ \oplus $ : additions
    \item $ \rhd $ : gains/multi
\end{itemize}
\textbf{Je comprend pas les boucles dans les circuits pfffff}

\subsection{Filtre Butterworts}
\begin{align*}
    \left| H(jw)^2 \right| &= \frac{1}{1+ (\frac{w}{w_c})^{2N}} \\
    \left| H(S) \right| &= \frac{w_c^N}{(S-p_1)(S-p_2)\dots(S-p_N)}\\
    R_p &= -10 \log_{10} \frac{1}{1 + (\frac{w_p}{w_c})^{2N}} \\
    A_s &= -10 \log_{10} \frac{1}{1 + (\frac{w_s}{w_c})^{2N}} \\
    \rightarrow N &= \left\lceil \frac{\log_{10} [(10^{R_p/10} - 1)/(10^{A_s/10} - 1)]}{2 \log_{10} \frac{w_p}{w_s}}\right\rceil \\
    w_{cp} &= \frac{w_p}{(10^{R_p/10} - 1)^{1/2N}} \\
    w_{cs} &= \frac{w_w}{(10^{A_p/10} - 1)^{1/2N}} \\
    \rightarrow w_c &= \frac{w_{cp} + w_{cs}}{2}
\end{align*}
N = nombre de pôle = nombre de racine. Si complexe, prendre uniquement partie réelle ou inverser la division $ \frac{\omega _p}{\omega _s} $ en $ \frac{\omega _s}{\omega _p} $ \\
Les pôles sont sur le cercle de rayon $ w_c $. La fonction de 
\begin{itemize}
    \item On prend $ 2N $ pôle mais on se serre que de la moiter : les stables à gauche de l'ordonnée. On divise donc $ \frac{2 \pi }{2N} $ ou $ \frac{360}{2N} $. Si $ N $ paire, penser à pivoter les pôles pour ne pas qu'il soit sur l'axe des ordonnés (conditionnellement stable)
    \item chaque pôle : $ p_i = w_c * e^{j * i \frac{2 \pi}{2N}} $ On prend que ceux au dessus de $ \pi /2 $ ou $ 90 $. On peut éventuellement simplifier avec la formule de l'exp complexe.
\end{itemize}


\section{Signaux discret}
\subsection{Transformé en Z}
Transformé en Z (fonction discrète) \begin{align*}
    X[Z] &= z\{x[n]\} \\ 
        &= \sum_{n=-\infty }^{+\infty }x[n]e^{j \Omega } \\
        &= \sum_{n=-\infty }^{+\infty }x[n]e^{j \omega T }
\end{align*}
On remplace $ \Omega  $ par $ \Omega = \omega T $ avec $ T = \frac{1}{Fs}$ la période = inverse de la fréquence d'échantillonnage.

\subsection{Pôles et zéros}
?????

\subsection{Réponse en fréquence}
Soit $ H[Z] $ une fonction de transfert, on remplace $ Z = e^{-j \Omega } = \cos \Omega - j \sin \Omega $ puis phase $ \left| H[Z] \right|  $ et gain $ \lfloor H[e^{j \Omega }] $ 
$\rightarrow$ Réponse en fréquence périodique par le $ \cos  $ et le $ \sin  $ 


\subsection{Stabilité}
Plot partie réel $ \mathcal{R} $ et partie imaginaire $ Im $ 
\begin{itemize}
    \item Un système T.D. est stable si $\rightarrow$ touts les pôles sont à l'intérieur du cercle unitaire.
    \item Un système T.D. est instable si \begin{itemize}
        \item Au moins un pôle est à l'extérieur du cercle unitaire
        \item Il existe des pôles multiples sur le cercle unitaire
    \end{itemize}
    \item Un système T.D. est conditionnellement stable si $\rightarrow$ il existe des pôles simples sur le cercle unitaire.
\end{itemize}

\subsection{Type de filtre}
On doit regarder $ \left| H[Z] \right| = \left| H[e^{j \Omega }] \right| = \left| H[e^{2 \pi f}] \right| $ qui serait caractérisé par la distance des pôles (et zéros ?) à un point sur le cercle dans le plans. On regarde le graphique de l'argument de ce point en abscise et du module en ordonné.

J'ai l'impression on regarde souvent par rapport à $ \pi  $ 

\textbf{Pourquoi il met $ \left| H(e^{j \Omega }) \right|  = \frac{r_1\dots r_M}{d_1\dots d_M} = \frac{1}{d_1}$ ??? } Si la phase est constante y'a pas de filtre ? Il doit y avoir un problème avec ce calcul

\subsection{Filtre IIR :}
\subsubsection{Filtre IIR : Transformation bilinéaire}
On passe d'un truc discret pour retourner vers le continue où on sait faire 
\begin{align*}    
    H(z) = H(S) \text{ avec } S = \frac{2}{T} \frac{1 - Z^{-1}}{1 + Z^{-1}}
    \Leftrightarrow Z = \frac{ S/2 + 1}{1 - S/2}
\end{align*}
$\rightarrow$ Permet de transformer nos pôles continue en pôles discret. On peut alors les dessiner dans le plans complexe avec le cercle unitaire et remarquer que c'est stable si dans le cercle.
    Où $ Z = e^{j \omega } $ 
\[
    \frac{2}{T} \tan \frac{\Omega }{2} = w \text{ pour les deux }\omega _s, \omega _r
.\]
        
\subsection{Filtre FIR : fenêtrage}
Classiquement on aura $ \Omega _s - \Omega _p = \frac{c*\pi }{M} $ avec $ M $ le nombre de coefficient à déterminer, et une attenuation \textbf{min} hors bande $ A_s \min  $ associé.

Les coefficients sont symétrique, donc la phase du filtre est linéaire ($ \lfloor H(e^{j \Omega }) = -2 \Omega $ )

Dans ce cas, l'ordre du filtre est $ M - 1 $.

\subsubsection{Type de filtre FIR}
Voir graphique sur OneNote pour la représentation des différents types c'est intéressant.
\begin{itemize}
    \item \textbf{Type 1:} On peut réaliser tous les types de filtres passe-bas, passe-bande, passe-haut
    \item \textbf{Type 2:} Il existe un zéro à $ \Omega = \pi  $ $\rightarrow$ On ne peut pas réaliser un filtre passe-haut
    \item \textbf{Type 3:} Il existe un zéro à $ \Omega = 0 $ et à $ \Omega = \pi  $ $\rightarrow$ One ne peut pas réaliser un filtre passe-bas, ni un filtre passe-haut.
    \item \textbf{Type 4:} Il existe un zéro à $ \Omega = 0 $ $\rightarrow$ on ne peut pas réaliser un filtre passe-bas
\end{itemize}
\begin{table}[!ht]
    \centering
    \begin{tabular}{|l|l|l|}
    \hline
        Symétrie de $h[n]$ & Nombre de coefficients $M$ & Type \\ \hline
        Symétrie positive $ h_n = - h_{M-1-n} $ & Impaire & 1 \\ \hline
        ~ & Paire & 2 \\ \hline
        Symétrie négative : $ h_n = - h_{M-1-n} $  & Impaire & 3 \\ \hline
        ~ & Paire & 4 \\ \hline
    \end{tabular}
\end{table}

\end{document}