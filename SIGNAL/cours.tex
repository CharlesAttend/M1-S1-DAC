\documentclass{article}
\usepackage[utf8]{inputenc}
\usepackage[a4paper, margin=2.5cm]{geometry}
\usepackage{graphicx}
\usepackage[french]{babel}

\usepackage[default,scale=0.95]{opensans}
\usepackage[T1]{fontenc}
\usepackage{amssymb} %math
\usepackage{amsmath}
\usepackage{amsthm}
\usepackage{systeme}

\usepackage{hyperref}
\hypersetup{
    colorlinks=true,
    linkcolor=blue,
    filecolor=magenta,      
    urlcolor=cyan,
    pdftitle={SIGNAL2},
    }
\urlstyle{same} %\href{url}{Text}

\theoremstyle{plain}% default
\newtheorem{thm}{Théorème}[section]
\newtheorem{lem}[thm]{Lemme}
\newtheorem{prop}[thm]{Proposition}
\newtheorem*{cor}{Corollaire}
%\newtheorem*{KL}{Klein’s Lemma}

\theoremstyle{definition}
\newtheorem{defn}{Définition}[section]
\newtheorem{exmp}{Exemple}[section]
% \newtheorem{xca}[exmp]{Exercise}

\theoremstyle{remark}
\newtheorem*{rem}{Remarque}
\newtheorem*{note}{Note}
%\newtheorem{case}{Case}



\title{Cours SIGNAL}
\author{Charles Vin}
\date{S1-2022}

\begin{document}
\maketitle

\underline{Nouveau cours du 21/10} \\

Super prof, super diapo. On a un poly avec les diapos et un poly avec les exo de TD.

Exam : Feuille A4 manuscrite \textbf{recto}. Attention démonstration de formules en exam. Faites uniquement en cours $\rightarrow$ Privilégie les gens qui viennent en cours + veux qu'on comprenne les math.

\section{Signaux et système}
\subsection{Généralité}

\underline{Nouveau cours du 28/10} \\
Again note sur le poly.

\textbf{J'ai demandé pour avoir le diapo et l'annoté numériquement mais y'a des choses issus de livre payant donc ça passe pas trop}. Il y aura donc plus rien ici lol

\underline{Nouveau cours du 25/11} \\
Je vais essayer de noter les démonstrations ici. Mais je pense qu'il vas rester des indications dans le poly.
\subsubsection{Diapo 33}
\begin{proof}[Preuve Changement d'échelle ]
\begin{align*}
    TF\{x(\alpha t)\} &= \int_{-\infty }^{+\infty x(\alpha t) e^{-j2 \pi ft}}dt \\
        \text{Changement de variable }t ' = \alpha t &\Leftrightarrow t = \frac{1}{\alpha }t' ; \frac{dt}{dt ' } = \frac{1}{\alpha } \\
        &= \begin{cases}
            \int_{t ' = - \infty }^{t ' = + \infty } x(t ' )e^{-j 2 \pi f \frac{t ' }{\alpha }} &\text{ si } \alpha > 0\\
            \int_{t ' = + \infty }^{t ' = - \infty } x(t ' )e^{-j 2 \pi f \frac{t ' }{\alpha }} &\text{ si } \alpha < 0\\
        \end{cases} \\
        &= \begin{cases}
            \frac{1}{\alpha } \int_{-\infty }^{+\infty }x(t ' ) e^{-j 2 \pi (\frac{f}{\alpha } t ' )} &\text{ si } \alpha > 0\\
            - \frac{1}{\alpha } \int_{-\infty }^{+\infty }x(t ' ) e^{-j 2 \pi (\frac{f}{\alpha } t ' )} &\text{ si } \alpha < 0\\
        \end{cases} 
\end{align*}

\begin{proof}[Preuve : Modulation ]
    \begin{align*}
        TF \{x(t) e^{j 2 \pi f_0 t}\} &= \int_{-\infty }^{+\infty }x(t) e^{j 2 \pi f_0 t} e^{-j 2 \pi f t}dt \\
            &= \int_{_\infty}^{+\infty }x(t) e^{-j 2 \pi (f - f_0)t}dt \\
            &= X(f - f_0)
    \end{align*}
\end{proof}

\subsubsection{Diapo 38}
\begin{align*}
    TF^{-1}\{\delta (f - f_0)\} &= \int_{-\infty }^{+\infty }\delta (f - f_0) e^{+j2 \pi ft}df \\
    &= \int_{-\infty }^{+\infty }\delta (f - f_0) e^{+j2 \pi f_0 t}df \\
    &= e^{+j2 \pi f_0 t} \int_{-\infty }^{+\infty }\delta (f - f_0) df \\
    &= e^{+j2 \pi f_0 t}
\end{align*}

\end{proof}


\end{document}