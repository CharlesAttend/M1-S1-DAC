\documentclass{article}
\usepackage[utf8]{inputenc}
\usepackage[a4paper, margin=2.5cm]{geometry}
\usepackage{graphicx}
\usepackage[french]{babel}

\usepackage[default,scale=0.95]{opensans}
\usepackage[T1]{fontenc}
\usepackage{amssymb} %math
\usepackage{amsmath}
\usepackage{amsthm}
\usepackage{systeme}

\usepackage{hyperref}
\hypersetup{
    colorlinks=true,
    linkcolor=blue,
    filecolor=magenta,      
    urlcolor=cyan,
    pdftitle={SIGNAL2},
    }
\urlstyle{same} %\href{url}{Text}

\theoremstyle{plain}% default
\newtheorem{thm}{Théorème}[section]
\newtheorem{lem}[thm]{Lemme}
\newtheorem{prop}[thm]{Proposition}
\newtheorem*{cor}{Corollaire}
%\newtheorem*{KL}{Klein’s Lemma}

\theoremstyle{definition}
\newtheorem{defn}{Définition}[section]
\newtheorem{exmp}{Exemple}[section]
% \newtheorem{xca}[exmp]{Exercise}

\theoremstyle{remark}
\newtheorem*{rem}{Remarque}
\newtheorem*{note}{Note}
%\newtheorem{case}{Case}



\title{Cours SIGNAL}
\author{Charles Vin}
\date{S1-2022}

\begin{document}
\maketitle

\underline{Nouveau cours du 21/10} \\

Super prof, super diapo. On a un poly avec les diapos et un poly avec les exo de TD.

Exam : Feuille A4 manuscrite \textbf{recto}. Attention démonstration de formules en exam. Faites uniquement en cours $\rightarrow$ Privilégie les gens qui viennent en cours + veux qu'on comprenne les math.

\section{Signaux et système}
\subsection{Généralité}

\underline{Nouveau cours du 28/10} \\
Again note sur le poly.

\textbf{J'ai demandé pour avoir le diapo et l'annoté numériquement mais y'a des choses issus de livre payant donc ça passe pas trop}. Il y aura donc plus rien ici lol

\underline{Nouveau cours du 25/11} \\
Je vais essayer de noter les démonstrations ici. Mais je pense qu'il vas rester des indications dans le poly.
\subsubsection{Diapo 33}
\begin{proof}[Preuve Changement d'échelle ]
\begin{align*}
    TF\{x(\alpha t)\} &= \int_{-\infty }^{+\infty x(\alpha t) e^{-j2 \pi ft}}dt \\
        \text{Changement de variable }t ' = \alpha t &\Leftrightarrow t = \frac{1}{\alpha }t' ; \frac{dt}{dt ' } = \frac{1}{\alpha } \\
        &= \begin{cases}
            \int_{t ' = - \infty }^{t ' = + \infty } x(t ' )e^{-j 2 \pi f \frac{t ' }{\alpha }} &\text{ si } \alpha > 0\\
            \int_{t ' = + \infty }^{t ' = - \infty } x(t ' )e^{-j 2 \pi f \frac{t ' }{\alpha }} &\text{ si } \alpha < 0\\
        \end{cases} \\
        &= \begin{cases}
            \frac{1}{\alpha } \int_{-\infty }^{+\infty }x(t ' ) e^{-j 2 \pi (\frac{f}{\alpha } t ' )} &\text{ si } \alpha > 0\\
            - \frac{1}{\alpha } \int_{-\infty }^{+\infty }x(t ' ) e^{-j 2 \pi (\frac{f}{\alpha } t ' )} &\text{ si } \alpha < 0\\
        \end{cases} 
\end{align*}

\begin{proof}[Preuve : Modulation ]
    \begin{align*}
        TF \{x(t) e^{j 2 \pi f_0 t}\} &= \int_{-\infty }^{+\infty }x(t) e^{j 2 \pi f_0 t} e^{-j 2 \pi f t}dt \\
            &= \int_{_\infty}^{+\infty }x(t) e^{-j 2 \pi (f - f_0)t}dt \\
            &= X(f - f_0)
    \end{align*}
\end{proof}

\subsubsection{Diapo 38}
\begin{align*}
    TF^{-1}\{\delta (f - f_0)\} &= \int_{-\infty }^{+\infty }\delta (f - f_0) e^{+j2 \pi ft}df \\
    &= \int_{-\infty }^{+\infty }\delta (f - f_0) e^{+j2 \pi f_0 t}df \\
    &= e^{+j2 \pi f_0 t} \int_{-\infty }^{+\infty }\delta (f - f_0) df \\
    &= e^{+j2 \pi f_0 t}
\end{align*}

\end{proof}

\underline{Nouveau cours du 02/12} \\

\subsubsection{Diapo 42}
La formule des coefs resemble beaucoup à celle de la corrélation entre $ x $ et $ e^{-j2 \pi nt/T} $. \\
En bas les sinus et cosinus ont des fréquences de $ n*\frac{t}{T} $ 

\subsubsection{Diapo 43}
Le $ a, a+T $ dans la formule du coeficient == on fait l'intégrale sur un motif période, si j'ai bien compris comme c'est égale à zéros partout, cette intégrale est équivalente à celle sur $ \pm \infty  $.

\subsubsection{Diapo 44}
Formule sympatique par exemple si on doit intégrer un $ sinc^2 $ ça revient à intégrer une fonction rectangle. \\
Densité spectrale : j'ai écrit un truc dans mon TD quand on a démontré la formule de Parseval. TD3 exo 1 je crois

\subsubsection{Diapo 45}
$ \gamma _x (\tau ) $ = autocorrelation 

\subsection{Approche fréquentielles des SLIs}
\subsubsection{Diapo 47}

\begin{defn}[]
    La \textbf{réponse en fréquence} d'un SLI 
    \[
        H = TF\{h\}= Y/X
    .\]
    car $ y = h \star x \Leftrightarrow Y = H*X$ 
\end{defn}
On dit que les SLI sont des filtres car avec $ Y=HX $ si $ H $ est proche de zéro ou très grand on vas supprimer ou amplifier certaine fréquence.

Le gain et la phase finalement on écrit $ H $ sous sa forme complexe.

\subsubsection{Diapo 48}

$ A_s(f_0) $ = amplitude de la sortie, $ A_e(f_0) $ = amplitude de l'entrée. $ \phi (f_0) $ = phase = décalage dans le temps. \\
On appele ça le gain car on retrouve que $ \left| H(f_0) \right| $  c'est le facteur qui change l'amplitude. De même pour la phase qui est le décalage, qui indique le déphasage.

Si j'injecte une sinusoïde en entrée, je peux retrouver en sortie le même signal mais en fonction du gain et de la phase.

Diagramme de bode : On utilise le $ \log_{10}(f)  $ en abscise pour représenter sur un même graph toutes les fréquences 

\subsubsection{Diapo 49}
Le transformé simplifie tout en virant les convolutions


\section{Numérisation et reconstitution des signaux}
= traitement numérique du signal 

Moitié des points sur cette partie, moitié des point sur le reste.
\subsection{Introduction}

\subsubsection{Diapo 52}
Quand on traite des signaux : \begin{itemize}
    \item Soit c'est pour les transmettre entre deux machines : et transmettre en numérique permet d'éviter les problèmes de bruit
    \item Soit c'est pour les manipuler numériquement sur des processeurs ect
\end{itemize}

\subsubsection{Diapo 53}
En pratique full signal analogique qu'on convertie h24 :
\begin{itemize}
    \item Convertisseur analogique - numérique (CAN) = échantillonneur
    \item CNA : blocage d'ordre zéro
\end{itemize}

\subsubsection{Diapo 54}
On discrétise l'abscisse mais aussi l'ordonnée. Discrétiser l'ordonnée provoque une erreur de quantification mais c'est nécessaire pour encoder en binaire $\rightarrow$ On met beaucoup de niveau quantifie pour éviter l'erreur.

\subsection{Échantillonnage idéal}
\subsubsection{Diapo 55}

idéal = une valeur à une date précise.

Comment choisir $ T_e $ ? On veut minimiser $ T_e $ tout en restant capable de reconstituer le signal. 

\subsubsection{Diapo 56}
Depuis le début, la transformé simplifie les choses. Regardons le lien entre $ x(t) $ et $ x[k] $ en fonction de $ T_e $ 
\begin{defn}[]
    \textbf{Signal échantilloné} = intermédiaire mathématique = signal continue d'une série d'impulsion de Dirac de hauteur modulé tout le $ T_e $ 
    \[
        x_e(t) = \sum_{k=-\infty }^{+\infty }x(kT_e)\delta (t-kT_e)
    .\]
    Relis $ x(t) $ et $ x[k] $.
\end{defn}

\subsubsection{Diapo 58}
On a écrit $ X_e(f) $ en fonction de $ X(f) $. On a le signal échantillonné en fonction du signal continue.

\subsubsection{Diapo 59}

\subsubsection{Diapo 60}
Annotation poly


\underline{Nouveau cours10 du 09/12} \\

Sur des signaux random, on défini une bande de fréquence à protéger, et au vas filtrer au delà (filtre anti aliasing). En fonction de la bande de fréquence qu'on conserve $ f_{max} $ on peut alors définir la fréquence d'échantillonnage max qui respecte le théorème de Shannon. Exemple diapo 63

\subsubsection{Diapo 65}
Parfois si on a du vide dans $ f_{max} $ il peut arriver qu'il n'y ait pas de recouvrement spectral même sans respecter Shannon (exo 2 TD4 je crois ) == Shannon condition suffisante mais pas nécéssaire.

Il faut être sur de pouvoir caler un nombre impaire de motif entre $ -f_0 $ et $ f_0 $, pour cela on défini un $ B' $ 

\subsection{Quantification}
\subsubsection{Diapo 66}
Quantification : division uniforme en plusieurs niveau quantifié entre le max et le min du signal. Si on peut on utilise un nombre de niveau $ D $ en puissance de deux pour pouvoir facilement convertir en binaire. 

Mais on introduit une erreur de quantification $ \epsilon _q = x[k] - x_q[k]$. Finalement on a deux erreurs introduite pour numériser : le cut de fréquence et la quantification

\subsection{Reconstitution par blocage}
\subsubsection{Diapo 67}
BUT : reconstituer un signal continue depuis un signal échantillonné. \\
C'est juste transformer un signal échantillonné en un signal en escalier. On utilise bien $ T_e $ entre deux marches $ [kT_e ; (k+1)T_e] $  

\subsubsection{Diapo 68}
oklm
\subsubsection{Diapo 69}
On part de l'exemple diapo 61. le module de $ P(f) $ c'est le $ sinc $ et $ T_e $. Le $ sinc $ s'annule en $ 1/T_e = f_e $. On retrouve finalement presque l'originale $ |X(f)| $ mais avec du bruit sur les cotés.
\subsubsection{Diapo 70}
Bruit qu'on vas filtrer toujours dans les fréquences

\subsubsection{Diapo 71}
Finalement $ x(t) $ est reconstitué par TF inverse. Avec le filtre passe bas, on arrondie les bords : car les marches d'escalier sont comme des sinusoïde haute fréquence, qu'on a coupé avec le filtre.

Autre remarque : Quand $ f_e $ est grand, au voisinage de 0 le sinc est vraiment proche de 1 sur une grande bande permettant une meilleurs reconstitution de $ \left| X(f) \right|  $ et de même le bruit lui est fort compressé. De plus si $ f_e $ grand devant $ f_{max} $, le filtre passe bas sera plus facile à réaliser (moins de coef) car on a plus de place pour faire chuter le signal

Problème : on veut aussi $ f_e $ petit d'après ce qu'on a dit avant, je sais plus pourquoi. Je pense c'est pour éviter d'avoir trop d'échantillon alors qu'on peut reconstituer sans

\subsection{Réduction de cadence et décimation}
\subsubsection{Diapo 72}
Réduction de cadence = échantillonnage sur un signal déjà à temps discret. Je prend un échantillons tous les $ N $. 
\[
    x^{\downarrow }[n] = x[k] \text{ avec } k = nN
.\]
Comment choisir $ N $ ? Il faut regarder l'effet dans le domaine des fréquences

\subsubsection{Diapo 74}
Exactement le même diapo 56 sur l'échantillonnage tellement c'est proche = transposition en numérique de l'échantillonnage, on vas encore introduire une intermédiaire mathématique. \textbf{Lors de la relecture du cours, mettre en parallèle l'échantillonnage et la réduction de cadence}

\subsubsection{Diapo 76}
Vraiment same que pour l'échantillonnage lol avec $ N T_e $ 

\subsubsection{Diapo 78}
Exemple de comment on fait avec des dessins pour trouver $ \left| X^{\downarrow }(v') \right|  $ avec $ N = 3 $. C'est la même chose, toujours attention à l'air des masses de dirac qui change

\subsubsection{Diapo 80}
Comme la dernière fois, on place un filtre anti-aliasing pour éviter un recouvrement spectrale dans la réduction de cadence == décimation

\subsection{Elévation de cadance et interpolation}
\subsubsection{Diapo 82}
Up sampling = rajouter des zéros entre les échantillons. On regarde si $ m $ est un multiple de $ k $ 
\[
    x^{\uparrow } [m] = \begin{cases}
    x[k] &\text{ pour } m = kM\\
    0 &\text{ pour } m \neq kM\\
    \end{cases} 
.\]
\subsubsection{Diapo 83}
Permet de lier le spectre avec et sans élévation de cadence.

\subsubsection{Diapo 84}
Cette fois ci on comprime l'axe des abscises

\subsubsection{Diapo 85}
Quel est l'intérêt de faire ça ? Recalculer les échantillons intermédiaires du à l'échantillonnage, c'est l'opération inverse du décimateur je crois. Voir TD5 exo 4. \\

\subsubsection{Diapo 86-87-88}
Démonstration que le décimateur c'est l'exact inverse de l'interpolateur

\subsubsection{Diapo 89}
Modification de cadence d'un facteur rationnel : Imaginons qu'au USA il utilise une autre norme pour échantillonner un CD que chez nous. Et qu'on souhaite lire ce CD en Europe.

Finalement ce dispositif peut changer la période d'échantillonnage comme on veux.

\subsubsection{Diapo 90}
On peux même combiner les deux filtres en un seul. Mais on peut avoir des pertes d'information. 

\subsubsection{Diapo 91}
On a vu 4 bloc de base \begin{itemize}
    \item Echantillonnage 
    \item Bloqueur d'ordre zéros
\end{itemize}
Equivalence en numérique \begin{itemize}
    \item Décimation
    \item Interpolation
\end{itemize}
Souvent on fait : Ech $\rightarrow$ Décimation $\rightarrow$ (on a graver le CD)$\rightarrow$ Interpolation $\rightarrow$ Blocage $\rightarrow$ (Lecture du CD).

Et on a toujours un filtre en amont d'un échantillonnage, et un filtre en aval pour un bloqueur.

\subsubsection{Diapo 92}
Exo 3 TD5 

\subsubsection{Diapo 93}
Exo 4 TD5 

Fin programme de l'exam














\end{document}