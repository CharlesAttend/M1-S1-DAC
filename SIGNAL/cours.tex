\documentclass{article}
\usepackage[utf8]{inputenc}
\usepackage[a4paper, margin=2.5cm]{geometry}
\usepackage{graphicx}
\usepackage[french]{babel}

\usepackage[default,scale=0.95]{opensans}
\usepackage[T1]{fontenc}
\usepackage{amssymb} %math
\usepackage{amsmath}
\usepackage{amsthm}
\usepackage{systeme}

\usepackage{hyperref}
\hypersetup{
    colorlinks=true,
    linkcolor=blue,
    filecolor=magenta,      
    urlcolor=cyan,
    pdftitle={SIGNAL2},
    }
\urlstyle{same} %\href{url}{Text}

\theoremstyle{plain}% default
\newtheorem{thm}{Théorème}[section]
\newtheorem{lem}[thm]{Lemme}
\newtheorem{prop}[thm]{Proposition}
\newtheorem*{cor}{Corollaire}
%\newtheorem*{KL}{Klein’s Lemma}

\theoremstyle{definition}
\newtheorem{defn}{Définition}[section]
\newtheorem{exmp}{Exemple}[section]
% \newtheorem{xca}[exmp]{Exercise}

\theoremstyle{remark}
\newtheorem*{rem}{Remarque}
\newtheorem*{note}{Note}
%\newtheorem{case}{Case}



\title{Cours SIGNAL}
\author{Charles Vin}
\date{S1-2022}

\begin{document}
\maketitle

\underline{Nouveau cours du 21/10} \\

Super prof, super diapo. On a un poly avec les diapos et un poly avec les exo de TD.

Exam : Feuille A4 manuscrite \textbf{recto}. Attention démonstration de formules en exam. Faites uniquement en cours $\rightarrow$ Privilégie les gens qui viennent en cours + veux qu'on comprenne les math.

\section{Signaux et système}
\subsection{Généralité}

\underline{Nouveau cours du 28/10} \\
Again note sur le poly.

\textbf{J'ai demandé pour avoir le diapo et l'annoté numériquement mais comme il veux qu'on viennent en cours : nop}. Il y aura donc plus rien ici lol


\end{document}