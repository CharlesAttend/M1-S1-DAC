\documentclass{article}
\usepackage[utf8]{inputenc}
\usepackage[a4paper, margin=2.5cm]{geometry}
\usepackage{graphicx}
\usepackage[french]{babel}

\usepackage[default,scale=0.95]{opensans}
\usepackage[T1]{fontenc}
\usepackage{amssymb} %math
\usepackage{amsmath}
\usepackage{amsthm}
\usepackage{systeme}

\usepackage{hyperref}
\hypersetup{
    colorlinks=true,
    linkcolor=blue,
    filecolor=magenta,      
    urlcolor=cyan,
    pdftitle={SIGNAL2},
    }
\urlstyle{same} %\href{url}{Text}

\theoremstyle{plain}% default
\newtheorem{thm}{Théorème}[section]
\newtheorem{lem}[thm]{Lemme}
\newtheorem{prop}[thm]{Proposition}
\newtheorem*{cor}{Corollaire}
%\newtheorem*{KL}{Klein’s Lemma}

\theoremstyle{definition}
\newtheorem{defn}{Définition}[section]
\newtheorem{exmp}{Exemple}[section]
% \newtheorem{xca}[exmp]{Exercise}

\theoremstyle{remark}
\newtheorem*{rem}{Remarque}
\newtheorem*{note}{Note}
%\newtheorem{case}{Case}



\title{Cours SIGNAL}
\author{Charles Vin}
\date{S1-2022}

\begin{document}
\maketitle

\underline{Nouveau cours du 21/10} \\

Super prof, super diapo. On a un poly avec les diapos et un poly avec les exo de TD.

Exam : Feuille A4 manuscrite \textbf{recto}. Attention démonstration de formules en exam. Faites uniquement en cours $\rightarrow$ Privilégie les gens qui viennent en cours + veux qu'on comprenne les math.

\section{Signaux et système}
\subsection{Généralité}

\underline{Nouveau cours du 28/10} \\
Again note sur le poly.

\textbf{J'ai demandé pour avoir le diapo et l'annoté numériquement mais y'a des choses issus de livre payant donc ça passe pas trop}. Il y aura donc plus rien ici lol

\underline{Nouveau cours du 25/11} \\
Je vais essayer de noter les démonstrations ici. Mais je pense qu'il vas rester des indications dans le poly.
\subsubsection{Diapo 33}
\begin{proof}[Preuve Changement d'échelle ]
\begin{align*}
    TF\{x(\alpha t)\} &= \int_{-\infty }^{+\infty x(\alpha t) e^{-j2 \pi ft}}dt \\
        \text{Changement de variable }t ' = \alpha t &\Leftrightarrow t = \frac{1}{\alpha }t' ; \frac{dt}{dt ' } = \frac{1}{\alpha } \\
        &= \begin{cases}
            \int_{t ' = - \infty }^{t ' = + \infty } x(t ' )e^{-j 2 \pi f \frac{t ' }{\alpha }} &\text{ si } \alpha > 0\\
            \int_{t ' = + \infty }^{t ' = - \infty } x(t ' )e^{-j 2 \pi f \frac{t ' }{\alpha }} &\text{ si } \alpha < 0\\
        \end{cases} \\
        &= \begin{cases}
            \frac{1}{\alpha } \int_{-\infty }^{+\infty }x(t ' ) e^{-j 2 \pi (\frac{f}{\alpha } t ' )} &\text{ si } \alpha > 0\\
            - \frac{1}{\alpha } \int_{-\infty }^{+\infty }x(t ' ) e^{-j 2 \pi (\frac{f}{\alpha } t ' )} &\text{ si } \alpha < 0\\
        \end{cases} 
\end{align*}

\begin{proof}[Preuve : Modulation ]
    \begin{align*}
        TF \{x(t) e^{j 2 \pi f_0 t}\} &= \int_{-\infty }^{+\infty }x(t) e^{j 2 \pi f_0 t} e^{-j 2 \pi f t}dt \\
            &= \int_{_\infty}^{+\infty }x(t) e^{-j 2 \pi (f - f_0)t}dt \\
            &= X(f - f_0)
    \end{align*}
\end{proof}

\subsubsection{Diapo 38}
\begin{align*}
    TF^{-1}\{\delta (f - f_0)\} &= \int_{-\infty }^{+\infty }\delta (f - f_0) e^{+j2 \pi ft}df \\
    &= \int_{-\infty }^{+\infty }\delta (f - f_0) e^{+j2 \pi f_0 t}df \\
    &= e^{+j2 \pi f_0 t} \int_{-\infty }^{+\infty }\delta (f - f_0) df \\
    &= e^{+j2 \pi f_0 t}
\end{align*}

\end{proof}

\underline{Nouveau cours du 02/12} \\

\subsubsection{Diapo 42}
La formule des coefs resemble beaucoup à celle de la corrélation entre $ x $ et $ e^{-j2 \pi nt/T} $. \\
En bas les sinus et cosinus ont des fréquences de $ n*\frac{t}{T} $ 

\subsubsection{Diapo 43}
Le $ a, a+T $ dans la formule du coeficient == on fait l'intégrale sur un motif période, si j'ai bien compris comme c'est égale à zéros partout, cette intégrale est équivalente à celle sur $ \pm \infty  $.

\subsubsection{Diapo 44}
Formule sympatique par exemple si on doit intégrer un $ sinc^2 $ ça revient à intégrer une fonction rectangle. \\
Densité spectrale : j'ai écrit un truc dans mon TD quand on a démontré la formule de Parseval. TD3 exo 1 je crois

\subsubsection{Diapo 45}
$ \gamma _x (\tau ) $ = autocorrelation 

\subsection{Approche fréquentielles des SLIs}
\subsubsection{Diapo 47}

\begin{defn}[]
    La \textbf{réponse en fréquence} d'un SLI 
    \[
        H = TF\{h\}= Y/X
    .\]
    car $ y = h \star x \Leftrightarrow Y = H*X$ 
\end{defn}
On dit que les SLI sont des filtres car avec $ Y=HX $ si $ H $ est proche de zéro ou très grand on vas supprimer ou amplifier certaine fréquence.

Le gain et la phase finalement on écrit $ H $ sous sa forme complexe.

\subsubsection{Diapo 48}

$ A_s(f_0) $ = amplitude de la sortie, $ A_e(f_0) $ = amplitude de l'entrée. $ \phi (f_0) $ = phase = décalage dans le temps. \\
On appele ça le gain car on retrouve que $ \left| H(f_0) \right| $  c'est le facteur qui change l'amplitude. De même pour la phase qui est le décalage, qui indique le déphasage.

Si j'injecte une sinusoïde en entrée, je peux retrouver en sortie le même signal mais en fonction du gain et de la phase.

Diagramme de bode : On utilise le $ \log_{10}(f)  $ en abscise pour représenter sur un même graph toutes les fréquences 

\subsubsection{Diapo 49}
Le transformé simplifie tout en virant les convolutions


\section{Numérisation et reconstitution des signaux}
= traitement numérique du signal 

Moitié des points sur cette partie, moitié des point sur le reste.
\subsection{Introduction}

\subsubsection{Diapo 52}
Quand on traite des signaux : \begin{itemize}
    \item Soit c'est pour les transmettre entre deux machines : et transmettre en numérique permet d'éviter les problèmes de bruit
    \item Soit c'est pour les manipuler numériquement sur des processeurs ect
\end{itemize}

\subsubsection{Diapo 53}
En pratique full signal analogique qu'on convertie h24 :
\begin{itemize}
    \item Convertisseur analogique - numérique (CAN) = échantillonneur
    \item CNA : blocage d'ordre zéro
\end{itemize}

\subsubsection{Diapo 54}
On discrétise l'abscisse mais aussi l'ordonnée. Discrétiser l'ordonnée provoque une erreur de quantification mais c'est nécessaire pour encoder en binaire $\rightarrow$ On met beaucoup de niveau quantifie pour éviter l'erreur.

\subsection{Échantillonnage idéal}
\subsubsection{Diapo 55}

idéal = une valeur à une date précise.

Comment choisir $ T_e $ ? On veut minimiser $ T_e $ tout en restant capable de reconstituer le signal. 

\subsubsection{Diapo 56}
Depuis le début, la transformé simplifie les choses. Regardons le lien entre $ x(t) $ et $ x[k] $ en fonction de $ T_e $ 
\begin{defn}[]
    \textbf{Signal échantilloné} = intermédiaire mathématique = signal continue d'une série d'impulsion de Dirac de hauteur modulé tout le $ T_e $ 
    \[
        x_e(t) = \sum_{k=-\infty }^{+\infty }x(kT_e)\delta (t-kT_e)
    .\]
    Relis $ x(t) $ et $ x[k] $.
\end{defn}

\subsubsection{Diapo 58}
On a écrit $ X_e(f) $ en fonction de $ X(f) $. On a le signal échantillonné en fonction du signal continue.

\subsubsection{Diapo 59}

\subsubsection{Diapo 60}
Annotation poly








\end{document}