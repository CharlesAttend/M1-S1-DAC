\documentclass{article}
\usepackage[utf8]{inputenc}
\usepackage[a4paper, margin=2.5cm]{geometry}
\usepackage{graphicx}
\usepackage[french]{babel}

\usepackage[default,scale=0.95]{opensans}
\usepackage[T1]{fontenc}
\usepackage{amssymb} %math
\usepackage{amsmath}
\usepackage{amsthm}
\usepackage{systeme}

\usepackage{hyperref}
\hypersetup{
    colorlinks=true,
    linkcolor=blue,
    filecolor=magenta,      
    urlcolor=cyan,
    pdftitle={Overleaf Example},
    % pdfpagemode=FullScreen,
    }
\urlstyle{same} %\href{url}{Text}

\theoremstyle{plain}% default
\newtheorem{thm}{Théorème}[section]
\newtheorem{lem}[thm]{Lemme}
\newtheorem{prop}[thm]{Proposition}
\newtheorem*{cor}{Corollaire}
%\newtheorem*{KL}{Klein’s Lemma}

\theoremstyle{definition}
\newtheorem{defn}{Définition}[section]
\newtheorem{exmp}{Exemple}[section]
% \newtheorem{xca}[exmp]{Exercise}

\theoremstyle{remark}
\newtheorem*{rem}{Remarque}
\newtheorem*{note}{Note}
%\newtheorem{case}{Case}



\title{Cours}
\author{Charles Vin}
\date{Date}

\begin{document}
\maketitle


\[
    x[k] \star \delta [k - k_0] = x[k-k_0]
.\]

Diapo 11 : 
\[
    f(t) \delta (t - t_0) = f(t_0) \delta (t - t_0)
.\]

Diapo 30 : def de la TF 
\[
    X(f) = TF\{x(t)\} = \int_{-\infty }^{+\infty }x(t)e^{-j2 \pi ft}dt
.\]
TF inverse 
\[
    x(t) = TF^{-1} \{X(f)\} = \int_{-\infty }^{+\infty } X(f) e^{+j2 \pi ft}dt
.\]


Diapo 31 : plusieurs prop sur la TF

Diapo 52 : Dérivé et TF
\[
    TF\{x^{(n)}\} = (j2 \pi f)^n X(f)
.\]
TF et convolution 

\[
    TF\{x(t) \star y(t)\} = X(f) * Y(f)
.\]

\[
    TF\{x(t) * y(t)\} = X(f) \star Y(f)
.\]



\end{document}