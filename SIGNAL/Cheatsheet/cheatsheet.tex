\documentclass{article}
\usepackage[utf8]{inputenc}
\usepackage[a4paper, margin=2.5cm]{geometry}
\usepackage{graphicx}
\usepackage[french]{babel}

\usepackage[default,scale=0.95]{opensans}
\usepackage[T1]{fontenc}
\usepackage{amssymb} %math
\usepackage{amsmath}
\usepackage{amsthm}
\usepackage{systeme}

\usepackage{hyperref}
\hypersetup{
    colorlinks=true,
    linkcolor=blue,
    filecolor=magenta,      
    urlcolor=cyan,
    pdftitle={},
    % pdfpagemode=FullScreen,
    }
\urlstyle{same} %\href{url}{Text}

\theoremstyle{plain}% default
\newtheorem{thm}{Théorème}[section]
\newtheorem{lem}[thm]{Lemme}
\newtheorem{prop}[thm]{Proposition}
\newtheorem*{cor}{Corollaire}
%\newtheorem*{KL}{Klein’s Lemma}

\theoremstyle{definition}
\newtheorem{defn}{Définition}[section]
\newtheorem{exmp}{Exemple}[section]
% \newtheorem{xca}[exmp]{Exercise}

\theoremstyle{remark}
\newtheorem*{rem}{Remarque}
\newtheorem*{note}{Note}
%\newtheorem{case}{Case}



\title{Cours}
\author{Charles Vin}
\date{Date}

\begin{document}

\section{Définition}
\begin{itemize}
    \item Formule d'Euler : $ e^{ix} = \cos x + i \sin x $ 
    \item Formule d'Euler : $ \cos x = \frac{e^{ix} + e^{-ix}}{2} $ et $ \sin x = \frac{e^{ix} - e^{-ix}}{2i} $
    \item Suite géométrique : $ \sum _{k=0}^{n}u_{k}=u_{0}+\cdots +u_{n}=u_{0}(1+q+\cdots +q^{n})=u_{0}{\frac {1-q^{n+1}}{1-q}}\ \ (q\neq 1) $
    \item trigo : $ \sin ^2(\theta ) = \frac{1 - \cos 2 \theta }{2}$, $ \cos ^2(\theta ) = \frac{1 + \cos 2 \theta}{2} $ 
    \item Peigne de Dirac de pas $ T $ $ W_T(t) = \sum_{k=-\infty }^{k=+\infty } \delta (t - kT) $
    \item Energie d'un signal à temps continue : $ E_x = \int_{\mathbb{R}}^{} x(t) x^*(t)dt = \int_{\mathbb{R}}^{} \left| x(t) \right| ^2 dt$ si $ x(t) \in \mathbb{R} $ 
    \item Si divergence : Puissance moyenne temps continue : $ P_x = \lim_{\theta  \to \infty} \frac{1}{2 \theta } x(t) x^*(t)dt $. Temps discret : $ \lim_{K \to \infty} \frac{1}{2K+1}\sum_{k=-K}^{K}x[k]x^*[k] $ 
    \item Intercorrélation temps continue : taux de ressemblance entre deux signaux décalés l'un par rapport à l'autre \begin{itemize}
        \item A energie finie : $ \gamma _{xy}(\tau ) = \int_{\mathbb{R}}^{}x(t) y^*(t - \tau ) dt$ 
        \item A puissance finie : $ \gamma _{xy}(\tau ) = \lim_{\theta  \to \infty} \frac{1}{2 \theta } \int_{\mathbb{R}}^{}x(t) y^*(t - \tau ) dt$ 
    \end{itemize} Discret reprendre le même schéma que au dessus
    \item Autocorrélation temps continue : taux de ressemblance avec une version décalée de lui-même. Même formule que l'intercorrélation mais avec $ y(t) = x(t) $ 
    \item Réponse impulsionnelle $ h(t) $ : on input $ \delta [k] $ dans notre SLI
    \item Réponse indicielle : on input $ u [k] $ dans notre SLI on obtient $ y_u(t) = \int_{-\infty }^{t}h(u)du $ fonction de répartition, et donc $ h $ densité
    \item Réponse en fréquence : $ H(f) = TF\{h(t)\} = Y / X $ 
    \item Gain de $ H $ = $ 2 \log_{10} \left| H(f) \right|  $, Phase : $ \psi(f) = \arg H(f) $ 
    \item Si SLI : $ y[k] = x[k] \star h[k] $ 
    \item Convolution : $ x(t) \star h(t) = \int_{\mathbb{R}}^{}h(\tau )x(t - \tau )d\tau $ 
    \item $ X(f) = TF\{x(t)\} = \int_{\mathbb{R}}^{} x(t) e^{-j 2 \pi ft}dt $ 
    \item $ x(t) = TF^{-1}\{X(f)\} = \int_{\mathbb{R}}^{} X(f) e^{+j 2 \pi ft}df $ 
    \item $ X(\nu ) = TF\{x[k]\} = \sum_{k = -\infty }^{+ \infty }x[k] e^{- j 2 \pi \nu k}$ périodique de période 1
    \item $ x[k] = TF^{-1} \{X(\nu )\} = \int_{0}^{1}X(\nu ) e^{+j 2 \pi \nu k} d \nu $
    \item Egalité de Plancherel : $ \int_{\mathbb{R}}^{}x_1(t)x_2^*(t) dt = \int_{\mathbb{R}}^{}X_1(f)X_2^*(f) df $ On peut intégré dans les deux domaines : $ \int_{}^{}sinc^2 \to \int_{}^{}rect  $ 
    \item Egalité de Parseval : $ \int_{\mathbb{R}}^{}\left| x(t) \right| ^2 dt = \int_{\mathbb{R}}^{}\left| X(f) \right| ^2df $ 
    \item Densité spectrale d'énergie (DSE) = Energie d'un signal calculé dans le domaine fréquentiel (voir thm de Wiener): $ \Gamma _x (f) = \left| X(f) \right| ^2 $ 
    \item Densité spectrale de puissance (DSP) = same = $ \Gamma _c (f) = \lim_{\theta  \to \infty} \frac{1}{2 \theta } \left| X_\theta (f) \right| ^2 $ avec $ X_\theta  $ DSE de $ x(t) $ limité à $ [-\theta ; \theta ] $ 
\end{itemize}


\section{Démonstration}
\begin{itemize}
    \item $ f(t) \delta(t - t_0) = f(t_0)\delta (t- t_0) $ 
    \begin{align*}
        \forall t \neq t_0 &: \delta (t - t_0) = 0 \text{ donc}\\
        &\Leftrightarrow f(t) \delta (t - t_0) = 0 \\
        &= f(t_0)\delta (t - t_0) \\
        \forall t = t_0 &: f(t) = f(t_0) \implies f(t) \delta (t - t_0) = f(t_0) \delta (t - t_0)
    \end{align*}
    \item $ \gamma _{xy} (\tau ) = \gamma _{yx} (-\tau ) $ faire un changement de variable $ t' = t - \tau  $ 
    \item $ x(t) \star \delta (t - \tau ) = x(t - \tau )$ définition into diamond formula 
    \item TF discrète de période 1 : montrer que $ X(\nu ) = X(\nu +1) $ 
    \item Propriété des TF : \begin{itemize}
        \item Retard : $ TF\{x(t - \tau )\} = e^{- j 2 \pi f \tau }X(f) $ changement de variable $ t' = t - \tau  $ 
        \item Inversion temporel : $ TF \{x(-t)\} = X(-f) $ changement de variable $ t' = - t  $ attention changement du sens des bornes
        \item Conjugaison : $ TF\{x^*(t)\} = X^*(f) = \int_{}^{}x^*(t) e^{-j 2 \pi ft} = (\int_{}^{}x(t) e^{j \pi ft})^* = (X(-f))^* $ 
        \item $ x $ réel + pair $ \implies TF\{x\} $ réel paire : $ x(-t) = x(t) \implies X(-t) = X(t) $ et $ x(t) = x^*(t) \implies X*(f) = X(f) $   
        \item Dérivation : $ TF\{x^{(n)}(t)\} = (j2 \pi f)^n X(f)$ et $ TF\{(-j2 \pi t)^n x(t)\} = X^{(n)}(f) $  
        \item Changement d'échelle : $ TF\{x(\alpha t)\} = \frac{1}{\left| \alpha  \right| }X(\frac{f}{\alpha }) $ : \textbf{compression de l'échelle du temps $ \implies  $ dilatation de l'échelle des fréquence} : changement de variable ATTENTION signe de $ \alpha  $ faire deux cas $ t' = \frac{t}{\alpha } $ 
        \item Modulation : $  TF \{x(t) e^{j 2 \pi f_0 t}\} = X(f - f_0) $ : évident 
        \item Convolution : $ TF\{x(t) \star y(t)\} =X(f)Y(f) $ : definition into Fubini into changement de var into cqfd
        \item Produit : Attention temps discret donne $ \circledast  $ : prouver par tf inverse 
    \end{itemize}
    \item Théorème de Wiener-Kintchine : $ \Gamma _x(f) = TF\{\gamma _x(\tau )\} $ : $ \gamma _x (\tau ) = x(\tau )\star x^*(\tau ) $ puis prop TF pour cqfd
\end{itemize}

\section{Système Linéaire, homogène et invariant}
\begin{itemize}
    \item Linéaire = Additivité + homogène 
    \item Additivité : \begin{itemize}
        \item Soit $ x_1(t) \to h(t) $ et $ x_2(t) \to h(t) $ $\rightarrow$ somme des deux sorties $ = y_1(t) $
        \item Soit $ x_1(t)+x_2(t) \to h(t) \to y_2(t) $
        \item Additif si $ y_1(t) = y_2(t) $  
    \end{itemize} 
    \item Homogène \begin{itemize}
        \item Soit $ x(t)*K \to h(t) \to y_1(t)$ 
        \item Soit $ x(t) \to h(t) \to *K \to y_2(t)$
        \item Homogène si $ y_1(t) = y_2(t) $
    \end{itemize}
    \item Invariance dans le temps \begin{itemize}
        \item Soit $ x(t) \to (t-T) \to h(t) \to y_1(t) $ 
        \item Soit $ x(t) \to h(t) \to t-T \to y_2(t) $ 
        \item Invariant si $ y_1(t) = y_2(t) $
    \end{itemize}
\end{itemize}

\section{Analyse en régime harmonique}
La réponse en fréquence d'un SLI peut être obtenue par une analyse en régime harmonique (régime sinusoïdale permanent) pour diverses valeurs de $ f_0 $ balayant l'axe des fréquences :
\[
    x(t) = A_e sin(2 \pi f_0 t) \to SLI \to y(t) = A_s(f_0) \sin (2 \pi f_0 t + \psi (f_0))
.\]
\[
    \left| H(f_0) \right| = \frac{A_s (f_0)}{A_e}
.\]
\[
    \psi (f_0) = arg H(f_0)
.\]

\section{Convolution à la main}
$ y(t) = x(t) \star h(t) = \int_{\mathbb{R}}^{}x(t-u)h(u)du $ 
\begin{enumerate}
    \item inversion du temps de $ x(u) \to x(-u) $
    \item Décalage en un $ t $ donné : $ x(-u) \to x(t - u) $
    \item Multiplication terme à terme avec $ h(u) : x(t - u) \to h(u)x(t-u) $  
    \item Intégration du produit sur $ \mathbb{R} $ 
\end{enumerate}
\section{TF}
Pour les TF complexe : 4 méthode : \begin{itemize}
    \item Brut force le calcul
    \item Penser à la formule de la dérivé 
    \item La fonction est le résultat d'une convolution
\end{itemize}
\begin{table}[!ht]
    \centering
    \begin{tabular}{|l|l|}
        \hline
        $x(t)$ & $X(f)$ \\ \hline
        $\delta(t)$ & 1 \\ \hline
        $\delta(t - t_0)$ & $e^{-j 2 \pi f t_0}$ \\ \hline
        1 & $\delta(t)$ \\ \hline
        $e^{-j 2 \pi f_0 t}$ & $\delta(f - f_0)$ \\ \hline
        $\cos (2 \pi f_0t)$ & $\frac{1}{2}\delta(f - f_0) + \frac{1}{2}\delta (f + f_0)$ \\ \hline
        $\sin (2 \pi f_0t)$ & $\frac{1}{2j}\delta(f - f_0) + \frac{1}{2j}\delta (f + f_0)$ \\ \hline
        $Rect_T (t)$ & $T sinc (\pi f T)$ \\ \hline
        $W_T(t)$ & $\frac{1}{T} \sum \delta(f - \frac{n}{T})$ \\ \hline
    \end{tabular}
\end{table}


\begin{table}[!ht]
    \centering
    \begin{tabular}{|l|l|}
    \hline
        $x[k]$ & $X(\nu)$ \\ \hline
        $\delta[k]$ & 1 \\ \hline
        $\delta[k - k_0]$ & $e^{-j 2 \pi \nu k_0}$ \\ \hline
        1 & $\delta(t)$ \\ \hline
        $e^{-j 2 \pi \nu_0 k}$ & $\delta(\nu - \nu_0)$ \\ \hline
        $\cos [2 \pi \nu_0t]$ & $\frac{1}{2}\delta(\nu -\nu_0) + \frac{1}{2}\delta (\nu +\nu_0)$ \\ \hline
        $\sin [2 \pi \nu_0t]$ & $\frac{1}{2j}\delta(\nu -\nu_0) + \frac{1}{2j}\delta (\nu +\nu_0)$ \\ \hline
        $x[k] = \begin{cases}
            1 & \forall k = 0, 1, \dots, N-1\\
            0 &\text{ sinon}\\
        \end{cases} $ & $\begin{cases}
        e^{-j \pi \nu (N-1) \frac{sin(\pi \nu N)}{sin(\pi \nu )}} &\text{ si } \nu \neq 0\\
        N                                                          &\text{ si } \nu  = 0\\
        \end{cases} $ \\ \hline
        $W_N[k]$ & $\frac{1}{N} \sum \delta(\nu - \frac{n}{N})$ \\ \hline
    \end{tabular}
\end{table}



\section{Échantillonnage}
$ x(t) \to x[k] = x(kT_e+), T_e $  période d'échantillonnage. \\
\begin{align*}
    x_e(t)  &= \sum_{k=-\infty }^{+\infty }x(kT_e)\delta (t - kT_e) \\
            &= \sum_{k=-\infty }^{+\infty }x[k]\delta (t - kT_e) \\
            &= x(t) \sum_{k}^{}\delta (t - kT_e)\\
    X_e(f)  &= \sum_{}^{} x[k] e^{- j 2 \pi fkT_e} \\
            &= X(\nu ) \vert _{\nu = fT_e = \frac{f}{f_e}} \\
            &= X(f) \star TF\{\sum_{}^{} \delta (t - kT_e)\} \\
            &= \frac{1}{T_e} \sum_{}^{}X(f - k \frac{1}{T_e}) \\
    X(\nu ) &= f_e \sum_{}^{}X(f - \frac{k}{T_e}) \vert _{f = \frac{\nu}{T_e} = \nu f_e}
\end{align*}
+ Filtre antialisaing pré-échantillonage: $ f_c = f_e / 2 $ pour éviter le recouvrement spectrale + \textbf{dessin}

\section{Blocage d'ordre zéro}
On veut : $ x[k] \to y(t) $. Inverse de l'échantillonnage : on cale des rectangles $ p(t) = rect_{T_e} (t - \frac{T_e}{2}) $ faire dessin
\begin{align*}
    y(t)&= \sum_{}^{}x[k]p(t - kT_e) \\
        &= \sum_{}^{}x[k](\delta (t - kT_e) \star p(t)) \\
        &= p(t) \star x_e(t) \\
    Y(f)&= P(f)X_e(f) \\
        &= T_e \frac{sin(\pi f T_e)}{\pi f T_e} e^{-j 2 \pi f T_e/2} * X_e(f)
\end{align*}
Le sinus cardinale s'annule en $ f = f_e $ et vaut $ 1/f_e $ en zéro. Il est proche de 1 autour de zéros. Par sa valeur en zéro, il annule l'effet du $ *f_e $ qu'on a après échantillonnage sur l'ordonnée\\
+ filtre post bloqueur en again $ f_c = f_e/2 $ pour retirer les ondelettes provoqué par le sinc autour de $ f_e $. 

\section{Réduction de cadence}
= échantillonnage version numérique = même démo que échantillonnage continue avec $ T_e = N $  : $ x[k] \to x^\downarrow [n] = x[k] = x[nN] $
\begin{align*}
    x_e[k] = \sum_{n}^{}x[nN] \delta (k-nN) \\
            &= \sum_{}^{}x^\downarrow [n] \delta (k - nN) \\
            &= x[k] \sum_{}^{}\delta (k - nN) \\
    X_e(\nu ) &= TF\{\sum_{}^{}x^\downarrow [n] \delta (k - nN)\} \\
            &= \sum_{}^{}x^\downarrow e^{- j 2 \pi \nu nN} \\
            &= X(\nu ') \vert_{\nu ' = \nu N} \\
    X_e(\nu ) &= X(\nu ) \circledast \frac{1}{N}\sum_{n= -\infty }^{+\infty }\delta (\mu - \frac{n}{N}) \\
            &= \frac{1}{N} X( \nu ) \star \sum_{n=0}^{N-1}\delta (\nu - \frac{n}{N}) \\
            &= \frac{1}{N} \sum_{}^{}X(\nu - \frac{n}{N}) \\
    X^\downarrow (\nu ') &= \frac{1}{N} \sum_{}^{}X(\nu - \frac{n}{N}) \vert _{\nu = \nu ' / N}\\
\end{align*}
Cette fois-ci on écarte l'axe des abscises $ \nu ' = \nu N $ . Shannon $ \frac{1}{N} > 2 \nu _{max} $  donc on filtre antialiasing $ f_c = \frac{1}{2N} $  avant 

\section{Élévation de cadence}
Insérer $ M - 1 $ zéros entre chaque point. \\
= bloqueur d'ordre zéros version numérique \\
$ x[k] \to x^\uparrow [m] = \begin{cases}
x[k] &\text{si } m=kM\\
0 &\text{sinon}\\
\end{cases}  $ \begin{align*}
    X^\uparrow (\nu '') &= TF \{x^\uparrow [m]\} = \sum_{m}^{}x^\uparrow [m] e^{-j 2 \pi \mu '' m} \\
    &= \sum_{k}^{}x[k] e^{-j 2 \pi \nu '' kM} \\
    &= X(\nu ) \vert _{\nu  = M \nu ''}
\end{align*}
On réduit l'axe des abscises par $ \nu '' = \nu / M $. \\
Filtre post traitement de $ f_c = \frac{1}{2M} $ \textbf{et de hauteur $ M $ } pour ré-équilibré le $ * \frac{1}{N} $ qu'on fait avec une réduction de cadence


\end{document}