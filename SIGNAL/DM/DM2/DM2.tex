\documentclass{article}
\usepackage[utf8]{inputenc}
\usepackage[a4paper, margin=2.5cm]{geometry}
\usepackage{graphicx}
\usepackage[french]{babel}

\usepackage[default,scale=0.95]{opensans}
\usepackage[T1]{fontenc}
\usepackage{amssymb} %math
\usepackage{amsmath}
\usepackage{amsthm}
\usepackage{systeme}

\usepackage{hyperref}
\hypersetup{
    colorlinks=true,
    linkcolor=blue,
    filecolor=magenta,      
    urlcolor=cyan,
    pdftitle={Compte Rendu TM2},
    % pdfpagemode=FullScreen,
    }
\urlstyle{same} %\href{url}{Text}

\theoremstyle{plain}% default
\newtheorem{thm}{Théorème}[section]
\newtheorem{lem}[thm]{Lemme}
\newtheorem{prop}[thm]{Proposition}
\newtheorem*{cor}{Corollaire}
%\newtheorem*{KL}{Klein’s Lemma}

\theoremstyle{definition}
\newtheorem{defn}{Définition}[section]
\newtheorem{exmp}{Exemple}[section]
% \newtheorem{xca}[exmp]{Exercise}

\theoremstyle{remark}
\newtheorem*{rem}{Remarque}
\newtheorem*{note}{Note}
%\newtheorem{case}{Case}



\title{Compte Rendu TM2}
\author{Charles Vin}
\date{30/09}

\begin{document}
\maketitle

\begin{enumerate}
    \item Trouver la transformée de Laplace inverse de \begin{enumerate}
        \item $ X(S) = \frac{10 (S+1)}{S^2 + 4S + 3} $ 
        \item $ X(S) = \frac{10 (S+1)e^{-2S}}{S^2 + 4S + 3} $ 
        \item $ X(S) = \frac{20}{S(S^2 + 10S + 16)} $ 
    \end{enumerate} 
    \item Tracer le diagramme de Bode de $ H(S) = \frac{20S(S+100)}{(S+2)(S+10)} $. Tracer la réponse en fréquence de $ H(S) $ en utilisant MatLab. Comparer les résultats \begin{align*}
        f&=1:0.001:1000\\
        w&=2 \pi * f \\
        S&=i*w \\
        H&=S \dot{} * (S+100) / \text{la fonction H du dessus} \\
        module &= 20 * \log_{10} (abs(H)) \\
        angle&(H) \\
        plot&(w, module) \\
        plot&(w, angle) \\
    \end{align*}
\end{enumerate}

\section{Exercice 1}
\begin{enumerate}
    \item Racine du polynôme : $ \{-1, -3\} $ 
    \begin{align*}
        X(S) &= \frac{10 (S+1)}{S^2 + 4S + 3} \\
            &= \frac{10 (S+1)}{(S+1)(S+3)} \\
            &= \frac{10}{S+3}\\
    \end{align*}
    D'après la ligne cinq du tableau on a 
    \[
        x(t) = e^\lambda u(t) \Leftrightarrow X(S) = \frac{1}{S - \lambda }
    .\]
    Dans notre cas $ \lambda = -3 $. Conclusion : 
    \begin{align*}
        X(S) &= \frac{10 (S+1)}{S^2 + 4S + 3} \\
        \downarrow \mathcal{L}^{-1}& \\
            x(t) &= 10e^{-3t}u(t)
    \end{align*} 
    
    \item \begin{align*}
        X(S) &= \frac{10 (S+1)e^{-2S}}{S^2 + 4S + 3} \\
        &= \frac{10 (S+1)e^{-2S}}{(S+1)(S+3)} \\
        &= \frac{10e^{-2S}}{(S+3)} 
    \end{align*}
    Nous somme dans le même cas que précédemment avec un time shifting de $ 2 $. Conclusion :
    \begin{align*}
        X(S) &= \frac{10 (S+1)e^{-2S}}{S^2 + 4S + 3}\\
        \downarrow \mathcal{L}^{-1}& \\
            x(t) &= 10e^{-3(t-2)}u(t-2)
    \end{align*}
    
    \item Les racines de $ S(S^2 + 10S + 16) $ sont $ \{0, -2, -8\} $ 
    \begin{align*}
        X(S) &= \frac{20}{S(S^2 + 10S + 16)} \\
            &= \frac{20}{S(S+2)(S+8)} \\
            &= \frac{A}{S} + \frac{B}{S+2} + \frac{C}{S+8} \\
    \end{align*}
    Trouvons $ A, B, C $ : 
    \begin{align*}
        &\frac{20}{S(S+2)(S+8)} = \frac{A}{S} + \frac{B}{S+2} + \frac{C}{S+8} \\
        \Leftrightarrow & 20 = A(S+2)(S+8) + B(S+8)S + C(S+2)S \\ 
        \Leftrightarrow & 20 = AS^2 + 10AS + 16A + BS^2 + 8BS + CS^2 + 2CS \\
        \Leftrightarrow & 20 = (A + B + C)S^2 + (10A + 8B + 2C)S + 16A
    \end{align*}
    Par identification : 
    \begin{align*}
        &\begin{cases}
            A + B + C = 0 \\
            10A + 8B + 2C = 0 \\
            16A = 20
        \end{cases} \\
        \Leftrightarrow &
        \begin{cases}
            B = -5/4 - C \\
            8(-5/4 - C) + 2C = -50/4 \\
            A = 20/16 = 5/4
        \end{cases} \\
        \Leftrightarrow &
        \begin{cases}
            B = -5/4 - C \\
            -6C = -50/4 + 40/4 = -10/4 \\
            A = 5/4 \\
        \end{cases} \\
        \Leftrightarrow &
        \begin{cases}
            B = -5/3 \\
            C = 5/12 \\
            A = 5/4 \\
        \end{cases}
    \end{align*}
    \begin{align*}
        X(S) &= \frac{20}{S(S^2 + 10S + 16)}\\
        \downarrow \mathcal{L}^{-1}& \\
        x(t) &= \frac{5}{4}u(t) - \frac{5}{3}e^{-2t}u(t) + \frac{5}{12}e^{-8t}u(t) \\
            &= u(t)(\frac{5}{4} - \frac{5}{3}e^{-2t} + \frac{5}{12}e^{-8t})
    \end{align*}
\end{enumerate}

\section{Exercice 2}
\begin{align*}
    H(S) &= \frac{20S(S+100)}{(S+2)(S+10)} \to _{p \to 0} \frac{20p*100}{2*20} = \frac{100p}{2} = 50p \\
    20 \log_{10} \left| H(jw) \right|  &= 20 \log_{10} \left| 50jw \right| = 20 \log_{10} 50w \\
    \text{Avec } w = 1 &: 20 \log_{10} \left| H(jw) \right| = 20 \log_{10} 50
\end{align*}
Grâce à ces calculs préalable, on peut maintenant tracer le diagramme de Bode du gain dans la figure \ref*{BodeGain}
\begin{figure}[htbp]
    \centering
    \includegraphics*[]{}
    \caption{Diagramme de Bode du gain}
    \label{BodeGain}
\end{figure}
\end{document}