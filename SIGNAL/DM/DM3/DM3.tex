\documentclass{article}
\usepackage[utf8]{inputenc}
\usepackage[a4paper, margin=2.5cm]{geometry}
\usepackage{graphicx}
\usepackage[french]{babel}

\usepackage[default,scale=0.95]{opensans}
\usepackage[T1]{fontenc}
\usepackage{amssymb} %math
\usepackage{amsmath}
\usepackage{amsthm}
\usepackage{systeme}

\usepackage{hyperref}
\hypersetup{
    colorlinks=true,
    linkcolor=blue,
    filecolor=magenta,      
    urlcolor=cyan,
    pdftitle={DM3 SIGNAL},
    % pdfpagemode=FullScreen,
    }
\urlstyle{same} %\href{url}{Text}

\theoremstyle{plain}% default
\newtheorem{thm}{Théorème}[section]
\newtheorem{lem}[thm]{Lemme}
\newtheorem{prop}[thm]{Proposition}
\newtheorem*{cor}{Corollaire}
%\newtheorem*{KL}{Klein’s Lemma}

\theoremstyle{definition}
\newtheorem{defn}{Définition}[section]
\newtheorem{exmp}{Exemple}[section]
% \newtheorem{xca}[exmp]{Exercise}

\theoremstyle{remark}
\newtheorem*{rem}{Remarque}
\newtheorem*{note}{Note}
%\newtheorem{case}{Case}



\title{DM3 SIGNAL}
\author{Charles Vin}

\begin{document}
\maketitle

\paragraph*{DM3 pour mercredi 12/10}
\subparagraph*{Exercice 1}
Avec $ H(S) = \frac{S^2 + w_0^2}{S^2 + 2Sw_0 \cos \theta + w_0^2} $. \begin{enumerate}
    \item Trouvez les pôles et les zéros de cette fonction de transfert
    \item Tracer les pôles et les zéros dans le plans complexe $ S $ 
    \item Quel type de filtre est réalisé par cette fonction de transfert ? Passe-bas, passe-haut, passe-bande, coupe-bande
    \item Avec MatLab tracez la réponse en fréquence de ce filtre pour \begin{enumerate}
        \item $ \theta = 60 \text{°}$
        \item $ \theta = 80 \text{°}$
        \item $ \theta = 87 \text{°}$
    \end{enumerate}
    et avec $ w_0 = 2 \pi f_0, f_0 = 50hz $ 
\end{enumerate}

\begin{enumerate}
    \item Trouvons les racines du dénominateur : $ S^2 + 2Sw_0\cos \theta  $ \begin{itemize}
        \item $ \Delta = (2w_0 \cos \theta )^2 - 4w_0^2  = 4 w_0^2 \cos ^2 \theta - 4 w_0^2 = 4w_0^2 ( \cos^2 \theta - 1)$ 
        \item $ \sqrt[]{\Delta } = 2 w_0 \sqrt[]{\cos ^2 \theta - 1} $ 
        \item Si $ w_0 = 0 $ ou si $ \theta = 0 $ alors $ \Delta = 0 $ il n'y a pas de racine.
        \item Sinon $ \forall w_0 \in R^*, \theta \in R^*, w_0^2 > 0, \cos ^2 (\theta )- 1 < 0 \Rightarrow \Delta < 0$ 
        \item Dans ce cas il y a deux racines complexes : $ \frac{-(2w_0 \cos \theta ) \pm i*2 w_0 \sqrt[]{\cos ^2 \theta - 1}}{2} = -w_0 \cos \theta \pm i * \sqrt[]{\cos ^2 \theta -1 }$ 
    \end{itemize} 
    \begin{align*}
        H(S) &= \frac{S^2 + w_0^2}{S^2 + 2Sw_0 \cos \theta + w_0^2} \\
        &= \frac{(S - iw_0)(S + iw_0)}
        {(S - w_0 \cos \theta \pm i * \sqrt[]{\cos ^2 \theta -1 }) (S + -w_0 \cos \theta \pm i * \sqrt[]{\cos ^2 \theta -1 })} \\
    \end{align*}
    On a donc deux pôles : $ \pm i*w_0(-\cos \theta + \sqrt[]{\cos^2 \theta - 1 }) $ et deux zéros $ \pm iw_0 $
    
    \item On pose $ w_0 = 2 \pi f_0, f_0 = 50hz $. Pour les coordonnes des points, voir la table \ref*{pt}. Pour le plot des points, voir la figure \ref{plot_pt}.\\
    On remarque que lorsque si $ \cos \theta  < 0 $ alors le système est instable. Si $ \cos \theta =0 $ le système est conditionnellement stable.
    \begin{table}[!ht]
        \centering
        \begin{tabular}{|l|l|l|l|l|}
        \hline
            $\theta$ & $\pi/2$ & $\pi/4$                                                & $\pi/3$                                           & $ 2\pi /3 $   \\ \hline
            $z_1$ & $(0,w_0)$   & $(0,w_0)$                                             & $(0,w_0)$                                         & $(0,w_0)$     \\ \hline
            $z_2$ & $(0, -w_0)$ & $(0, -w_0)$                                           & $(0, -w_0)$                                       & $(0, -w_0)$   \\ \hline
            $p_1$ & $(0, w_0)$  & $(-w_0*\frac{\sqrt[]{2}}{2}, \frac{w_0}{\sqrt[]{2}})$  & $(-\frac{w_0}{2}, \frac{\sqrt[]{3}}{2}w_0)$      & $(\frac{w_0}{2}, \frac{\sqrt[]{3}}{2}w_0)$ \\ \hline
            $p_2$ & $(0, -w_0)$ & $(-w_0*\frac{\sqrt[]{2}}{2}, -\frac{w_0}{\sqrt[]{2}})$  & $(-\frac{w_0}{2}, -\frac{\sqrt[]{3}}{2}w_0)$    & $(\frac{w_0}{2}, -\frac{\sqrt[]{3}}{2}w_0)$ \\ \hline
        \end{tabular}
        \caption{Coordonnées des pôles et des zéros pour $ w_0 = 2 \pi * 50 $}
        \label{pt}
    \end{table}
    \begin{figure}[htbp]
        \centering
        \includegraphics*[width=.75\textwidth]{./pole_zero.png}
        \caption{Pôle et zéros pour $ w_0 = 2 \pi * 50 $ }
        \label{plot_pt}
    \end{figure}


    \item Parcourons l'axe des imaginaires de $ jw = 0 \to +\infty  $. \begin{itemize}
        \item Lorsque $ w=0, \left| H(jw) \right| = 1$ car les $ p_i $ et $ z_i $ sont placé sur un cercle.
        \item Lorsque $ w=w_0, \left| H(jw) \right| = 0$ car on se place sur $ z_1 $ donc le numérateur s'annule  
        \item Lorsque $ w \to + \infty , \left| H(jw) \right| \to 1$ également.
    \end{itemize}
    On peut donc déduire que le gain se réduit vers zéros autour de $ w_0 = 2 \pi f_0 $ puis remonte vers un (on pourra le voir sur la figure MatLab dans la question suivante). C'est donc un filtre coupe bande sur $ f_0 $.
    
    \item Pour les figures, voir la page suivante.
\end{enumerate}

\end{document}