\documentclass{article}
\usepackage[utf8]{inputenc}
\usepackage[a4paper, margin=2.5cm]{geometry}
\usepackage{graphicx}
\usepackage[french]{babel}

\usepackage[default,scale=0.95]{opensans}
\usepackage[T1]{fontenc}
\usepackage{amssymb} %math
\usepackage{amsmath}
\usepackage{amsthm}
\usepackage{systeme}

\usepackage{hyperref}
\hypersetup{
    colorlinks=true,
    linkcolor=blue,
    filecolor=magenta,      
    urlcolor=cyan,
    pdftitle={DM3 SIGNAL},
    % pdfpagemode=FullScreen,
    }
\urlstyle{same} %\href{url}{Text}

\theoremstyle{plain}% default
\newtheorem{thm}{Théorème}[section]
\newtheorem{lem}[thm]{Lemme}
\newtheorem{prop}[thm]{Proposition}
\newtheorem*{cor}{Corollaire}
%\newtheorem*{KL}{Klein’s Lemma}

\theoremstyle{definition}
\newtheorem{defn}{Définition}[section]
\newtheorem{exmp}{Exemple}[section]
% \newtheorem{xca}[exmp]{Exercise}

\theoremstyle{remark}
\newtheorem*{rem}{Remarque}
\newtheorem*{note}{Note}
%\newtheorem{case}{Case}



\title{DM3 SIGNAL}
\author{Charles Vin}

\begin{document}
\maketitle

\paragraph*{DM3 pour mercredi 12/10}
\subparagraph*{Exercice 1}
Avec $ H(S) = \frac{S^2 + w_0^2}{S^2 + 2w_0 \cos \theta + w_0^2} $. \begin{enumerate}
    \item Trouvez les pôles et les zéros de cette fonction de transfert
    \item Tracer les pôles et les zéros dans le plans complexe $ S $ 
    \item Quel type de filtre est réalisé par cette fonction de transfert ? Passe-bas, passe-haut, passe-bande, coupe-bande
    \item Avec MatLab tracez la réponse en fréquence de ce filtre pour \begin{enumerate}
        \item $ \theta = 60 \text{°}$
        \item $ \theta = 80 \text{°}$
        \item $ \theta = 87 \text{°}$
    \end{enumerate}
    et avec $ w_0 = 2 \pi f_0, f_0 = 50hz $ 
\end{enumerate}


\end{document}