\documentclass{article}
\usepackage[utf8]{inputenc}
\usepackage[a4paper, margin=2.5cm]{geometry}
\usepackage{graphicx}
\usepackage[french]{babel}

\usepackage[default,scale=0.95]{opensans}
\usepackage[T1]{fontenc}
\usepackage{amssymb} %math
\usepackage{amsmath}
\usepackage{amsthm}
\usepackage{systeme}

\usepackage{hyperref}
\hypersetup{
    colorlinks=true,
    linkcolor=blue,
    filecolor=magenta,      
    urlcolor=cyan,
    pdftitle={Overleaf Example},
    % pdfpagemode=FullScreen,
    }
\urlstyle{same} %\href{url}{Text}

\theoremstyle{plain}% default
\newtheorem{thm}{Théorème}[section]
\newtheorem{lem}[thm]{Lemme}
\newtheorem{prop}[thm]{Proposition}
\newtheorem*{cor}{Corollaire}
%\newtheorem*{KL}{Klein’s Lemma}

\theoremstyle{definition}
\newtheorem{defn}{Définition}[section]
\newtheorem{exmp}{Exemple}[section]
\newtheorem{xca}[exmp]{Exercise}

\theoremstyle{remark}
\newtheorem*{rem}{Remarque}
\newtheorem*{note}{Note}
%\newtheorem{case}{Case}



\title{Cours}
\author{Charles Vin}
\date{Date}

\begin{document}
\maketitle

\underline{Nouveau TP du 22/09} \\

\begin{defn}[Parité]
    Fonction paire et impaire 
    \begin{itemize}
        \item si $ x(t) = x(-t) $ $\rightarrow$ fonction paire
        \item si $ x(t) = -x(t) $ $\rightarrow$ fonction impaire
    \end{itemize}
    On peut écrire une fonction $ x(t) $ de ses composantes paire et impaire.
    \[
        x(t) = \frac{1}{2}(x(t) + x(-t)) + \frac{1}{2}(x(t) - x(-t))
    .\]
\end{defn}


\begin{xca}[1]
    Écrivez $ x(t) = e^{-at}u(t)$ en fonction de ses composantes paires et impaires. \begin{itemize}
        \item La fonction paire $ x_p(t) = \frac{1}{2}(x(t) + x(-t)) $ 
        \item La fonction impaire $ \frac{1}{2} (x(t) - x(-t)) $ 
    \end{itemize}
    Dessin dans OneNote.
\end{xca}

\section{Système linéaire et invariant dans le temps}
\begin{defn}[Système linéaire et invariant dans le temps]
    Le système doit être \textbf{additif}, \textbf{homogène} et \textbf{invariant dans le temps}. \begin{itemize}
        \item Additif : Si $ y_1(t) = y_2(t) $ donc le système $ h(t) $ est additif.
        \item Homogène : Si $ y_1(t) = y_2(t) $ donc le système $ h(t) $ est homogène.
        \item Invariant dans le temps : Si $ y_1(t) = y_2(t) $ donc le système est invariant dans le temps.
        \item Voir dessin Onenote
    \end{itemize}
\end{defn}

\begin{xca}[2]
    Trouvez si les systèmes suivant sont linéaire et invariant dans le temps.\begin{align}
        y(t) = x(t)cos(2 \pi ft)\\
        y(t) = sin(x(t))
    \end{align}
    Réponse dans OneNote
\end{xca}

\begin{xca}[]
    Trouver le résultat de la convolution $ x(t) \star h(t) $ \begin{align*}
        x(t) &= u(t) - u(t-4) \\
        h(t) &= tu(t)
    \end{align*}
    Correction dans OneNote : \begin{align*}
        x(t) &= \int_{-\infty }^{\infty } x(\tau )h(t-\tau )d \tau \\
            &= \int_{-\infty }^{\infty } h(\tau )x(t-\tau )d \tau
    \end{align*}
\end{xca}

\begin{xca}[]
    Trouver le résultat de la convolution $ x(t) \star h(t) $ \begin{align*}
        x(t) &= u(t) - u(t-4) \\
        h(t) &= tu(t)
    \end{align*}
    Correction dans OneNote
\end{xca}

\begin{xca}[]
    Trouver le résultat de la convolution $ x(t) \star h(t) $ \begin{align*}
        x(t) &= \sin (t) [u(t) - u(t-2 \pi )] \\
        h(t) &= u(t-1) - u(t-3)
    \end{align*}
    Correction dans OneNote
\end{xca}

\paragraph*{A rendre sous forme de compte rendu avant le 28/09/2022}
\begin{xca}[]
    Trouver les composante paires et impaires de $ x(t) $: Dessin OneNote
\end{xca}
\begin{xca}[]
    Déterminez si les systèmes suivants sont linéaire et invariant dans le temps \begin{align}
        y(t) &= t^2 \frac{dx(t)}{dt} \\
        y(t) &= \cos (2 \pi ft + x(t))
    \end{align}
\end{xca}
\begin{xca}[]
    Trouver la convolution de $ x(t) $ et $ h(t) $ : 
    \begin{itemize}
        \item Dessin OneNote
        \begin{align*}
            x(t) &= 2 u(t-10) \\
            h(t) &= \sin (2t) u(t)
        \end{align*}
        \item \begin{align*}
            x(t) &= u(t) \\
            h(t) &=\text{ voir onenote}
        \end{align*}
        \item Voir OneNote
    \end{itemize}
\end{xca}
\begin{xca}[]
    Tracer avec MatLab les 3 courbes suivantes sur la même figure $ \forall t \in [0, 10] $ 
    \begin{align}
        w(t) &= e^{-t} \\
        x(t) &= te^{-t} \\
        y(t) &= e^{-t} + te^{-t}
    \end{align}
\end{xca}

\paragraph*{Intro MatLab}
Commande importante \begin{itemize}
    \item Size : check la size d'un vecteur 
    \item Length : Pareil pour les matrices
    \item Load/Save : Sauver des résultats
\end{itemize}
On évite $ i $ et $ j $ car dans MatLab c'est les variables complexes.\\
Faire attention au opérateur $ * $ = le produit scalaire et $ .* $  = le produit terme à terme de matrice.\\
Faire le graph de sin : \begin{align*}
    t &= 1:0.001:10 \\
    y &= sin(t) \\
    plot&(t,y) \\
    xlabel&('\text{Temps en seconde}') \\
    ylabel&('\text{Amplitude}')
\end{align*}
Log scale : $ semilogx(f, module) $, $ subplot $ pour subplot. \\
Plusieurs courbes : \begin{align*}
    x_1 &= e^{-t} \\
    x_2 &= t \\
    x_3 &= t + e^{-t} \\
    plot(t,x_1,'b', t,x_2,'r', t,x_3, 'g') \\
    c
\end{align*}




\end{document}