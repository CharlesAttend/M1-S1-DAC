\documentclass{article}
\usepackage[utf8]{inputenc}
\usepackage[a4paper, margin=2.5cm]{geometry}
\usepackage{graphicx}
\usepackage[french]{babel}

\usepackage[default,scale=0.95]{opensans}
\usepackage[T1]{fontenc}
\usepackage{amssymb} %math
\usepackage{amsmath}
\usepackage{amsthm}
\usepackage{systeme}

\usepackage{hyperref}
\hypersetup{
    colorlinks=true,
    linkcolor=blue,
    filecolor=magenta,      
    urlcolor=cyan,
    pdftitle={MAPSI},
    % pdfpagemode=FullScreen,
    }
\urlstyle{same} %\href{url}{Text}

\theoremstyle{plain}% default
\newtheorem{thm}{Théorème}[section]
\newtheorem{lem}[thm]{Lemme}
\newtheorem{prop}[thm]{Proposition}
\newtheorem*{cor}{Corollaire}
%\newtheorem*{KL}{Klein’s Lemma}

\theoremstyle{definition}
\newtheorem{defn}{Définition}[section]
\newtheorem{exmp}{Exemple}[section]
% \newtheorem{xca}[exmp]{Exercise}

\theoremstyle{remark}
\newtheorem*{rem}{Remarque}
\newtheorem*{note}{Note}
%\newtheorem{case}{Case}



\title{Cours: MAPSI}
\author{Charles Vin}
\date{2022}

\begin{document}
\maketitle
\underline{Nouveau cours du 13/09} \\
\section{Introduction}

\begin{itemize}
    \item Exam final : 50\%
    \item Partiel : 35\%
    \item Participation : 15\%
    \begin{itemize}
        \item travail dans la séance
        \item TME soumis en fin de séance omg 
    \end{itemize}
\end{itemize}

Deux grand type de modèle : \begin{itemize}
    \item Modèle paramétrique : connaissance sur la distribution stat des données. Puis on estime les paramètres de la loi.
    \item Modèle non paramétrique : l'inverse, on ne connait pas la loi. exemple : regression logistique
\end{itemize}

Echantillons : \begin{itemize}
    \item population
    \item ect 
\end{itemize}

\begin{defn}[]
    Vocabulaire :   
    \begin{itemize}
        \item Voir diapo 9/51
    \end{itemize}
\end{defn}

\begin{defn}[Mesure de proba]
    Une fonction qui associe chaque événement à une valeur entre 0 et 1.
    Voir diapo 15, definition importante.
\end{defn}

\begin{defn}[]
    \[
        P(A \cup B) = P(A) + P(B) - P(A \cap B)
    .\]
    Densité de proba 
    \[
        \text{Retrouver la définition}
    .\]
    
    Fonction de répartition
    \[
        F(x) = P(X<x) = \int_{-\infty }^{x}f(x)dx
    .\]
    Espérance : \begin{align*}
        E(X) &= \sum_{}^{}x_k*p_k \\
        E(X) &= \int_{}^{}X p(x) dx \\ 
        E(aX+b) = aE(X) + b \\
        E(X + Y) = E(X)+E(Y)
    \end{align*}

    Le Mode \begin{align*}
        p(Mo) = \max _k p(x_k)
        p(Mo) = \max _x p(x)
    \end{align*}

    Variance : \begin{align*}
        \sigma ^2 = \sum_{}^{}(x_k - E(X))^2 \\
        \sigma ^2 = \int_{}^{}(x-E(X))^2 p(x)dx \\
        V(aX+b) = a^2V(X) \\
        V(X) = E(X^2) - E(X)
    \end{align*}

    Médiane et quantile \begin{align*}
        \text{idk diapo}
    \end{align*}
\end{defn}

\begin{defn}[Loi marginale]
    La marginalisation consiste à projeter une loi jointe sur l'une des variables aléatoires. Par exemple extraire $ P(A) $ à partir de $ P(A,B) $. 
    \[
        P(A) = \sum_{i}^{}P(A, B = p b_i)
    .\]
    C'est la somme de la ligne ou de la colonne du tableau.
\end{defn}

\begin{defn}[]
    Probabilités conditionnelles
    \begin{align*}
        &P(A|B) = \frac{P(A \cup B)}{P(B)} \\ 
        \Leftrightarrow& P(A \cup B) = P(A|B)P(B)
    \end{align*}
    \begin{prop}[]
        \begin{itemize}
            \item Réversibilité : $ P(A,B) = P(A|B) $
            \item Théorème de Bayes : 
            \[
                P(A|B) = \frac{P(B|A)P(A)}{P(B)}
            .\]
            \item Intégration des probabilités totale
            \item DIAPO 39
        \end{itemize}
    \end{prop}    
    
\end{defn}

\begin{defn}[Indépendance probabiliste]
    Deux événements A et B sont indépendants si 
    \[
        P(A,B) = P(A) * P(B)
    .\]
    Corollaire : $ P(A|B) = P(A) $ 
\end{defn}

\begin{defn}[]
    La covariance 
    \[
        cov(X,Y) = E[(X-E(X))(Y-E(Y))]
    .\]
\end{defn}

\begin{defn}[Coefficient de corrélation linéaire ]
    Soit X,Y deux variables. Le coefficient de corrélation linéaire entre X et Y est:
    \[
        r = \frac{cov(X,Y)}{\sigma _X \sigma _Y}
    .\]
\end{defn}

CCL:
\begin{itemize} VOIR DIAPO
    \item Probabilité
    \item Marginalisation
    \item Conditionnement
    \item Indépendance : Si $ X_1 $ et $ X_2 $ sont indépendantes : $ P(X_1, X_2) = P(X_1)P(X_) $ 
\end{itemize}





\end{document}