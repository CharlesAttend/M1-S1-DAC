\documentclass{article}
\usepackage[utf8]{inputenc}
\usepackage[a4paper, margin=2.5cm]{geometry}
\usepackage{graphicx}
\usepackage[french]{babel}

\usepackage[default,scale=0.95]{opensans}
\usepackage[T1]{fontenc}
\usepackage{amssymb} %math
\usepackage{amsmath}
\usepackage{amsfonts}
\usepackage{amsthm}
\usepackage{systeme}

\usepackage{hyperref}
\hypersetup{
    colorlinks=true,
    linkcolor=blue,
    filecolor=magenta,      
    urlcolor=cyan,
    pdftitle={Fiche from TD},
    % pdfpagemode=FullScreen,
    }
\urlstyle{same} %\href{url}{Text}

\theoremstyle{plain}% default
\newtheorem{thm}{Théorème}[section]
\newtheorem{lem}[thm]{Lemme}
\newtheorem{prop}[thm]{Proposition}
\newtheorem*{cor}{Corollaire}
%\newtheorem*{KL}{Klein’s Lemma}

\theoremstyle{definition}
\newtheorem{defn}{Définition}[section]
\newtheorem{exmp}{Exemple}[section]
% \newtheorem{xca}[exmp]{Exercise}

\theoremstyle{remark}
\newtheorem*{rem}{Remarque}
\newtheorem*{note}{Note}
%\newtheorem{case}{Case}



\title{Ficher LRC à partir des TD}
\author{Charles Vin}
\date{M1-S1 2022}

\begin{document}
\maketitle

\section{Formule}
\begin{itemize}
    \item \textbf{$ F $ insatisfiable $ \Leftrightarrow \neg F$ valide}
    \item $ F $ satisfiable $ \Leftrightarrow \neg F$ non valide
    \item $ \neg F $ satisfiable $ \Leftrightarrow F$ non valide
    \item $ F $ valide $ \Leftrightarrow \neg F$ insatisfiable
    \item \textbf{$ A \to B \equiv \neg A \vee B $ }
    \item On développe $ \vee  $ comme un $ + $ et $ \wedge  $ comme un $ \times  $ 
\end{itemize}

\section{Méthode des tableaux}
\begin{itemize}
    \item S'entrainer ! TME1, exo 2
    \item Règle $ \alpha  $ = règles conjonctive, $ \beta  $ règles disjonctive = On sépare en deux branches
    \item On s'arrete lorsque full atome dans la boite
    \item Feuille fermé $ \Leftrightarrow $ contradiction entre atome, 
    \item Feuille ouverte = une solution, avec toutes les combinaisons possible qui respecte ce qu'il y a dans la feuille (penser à l'ensemble vide si tout est faux).
    \item \textbf{Si toute les feuilles de l'arbre sont fermées alors $ F $ unsat} $ \Leftrightarrow $ \textbf{Une feuille ouverte $\rightarrow F$ satisfiable}
    \item Indiquer quelle règle on utilise sur le coté.
\end{itemize}

\section{Système de Hilbert}
\subsection{Preuve dans Hilbert}
\begin{itemize}
    \item S'entrainer ! TME1, exo3-4
    \item On a : 3 axiomes + Modus Ponens
    \item Théorème de la déduction : $ A_1, \dots, A_n \models B \Leftrightarrow A_1, \dots, A_{n-1} \models A_n \to B$ 
    \item Les HP de départ sont déduite à partir du théorème de la déduction : on passe tout à gauche pour n'avoir plus qu'un atome à droite et bim voilà nos HP de départ. Par exemple : $ \models (P \to Q) \to ((Q \to R) \to (P \to R)) $ devient $ (P \to Q), (Q \to R), P \models R  $  
\end{itemize}

\subsection{Traduction d'énoncé en Hilbert}
\begin{itemize}
    \item Les $ \exists  $ n'aime pas les $ \rightarrow  $ à cause de $ A \to B \equiv \neg A \vee B $
\end{itemize}

\section{Logique du première ordre}
\begin{itemize}
    \item Définition d'un modèle : fonction($ \left| M \right| \to \left| M \right|  $ ) + prédicat ($ \left| M \right| \to  $ vrais/faux)
    \item Lors de la traduction d'énoncé en LPPO, c'est pas mal de garder des $ \forall, \exists  $ à l'intérieur de la clause, ça permet de la simplifier pas mal finalement
    \item Dans l'annale ils se sont pas fait chier sur ça, je cherchais compliqué alors que c'était simple 
\end{itemize}

\subsection{Preuve par résolution}
\begin{itemize}
    \item On ne peut simplifier qu'un truc à la fois : 
        \[
            \frac{\neg a \vee b \vee c \qquad a \vee \neg b \vee c }{b \vee \neg b \vee c}
        .\]
    \item Mieux de le faire en version Hilbert, permet de réutiliser les lignes plutôt que de les réécrire.
        \begin{align*}
            R_1 : R(c,d) \qquad [Res(C_2, C_4) ; \{Y\backslash d\}]
        \end{align*}
    \item \textbf{A refaire au moins une fois}
\end{itemize}

\subsection{Unification}
\begin{itemize}
    \item Unification : on peut changer les variables des deux cotés. 
    \item Filtrage : On ne peut changer que les variables de $ F_1 $ 
    \item Classiquement, on cherche $ F_2 = \sigma (F_1) $ avec $ \sigma = \{X/X', Z/g(a, X'), \dots \}$ un ensemble de substitution.
    \item Utiliser des $ X \prime  $ pour pas se tromper
    \item Écrire les variables en majuscule, et les constante en minuscule !
    \item On ne peux pas changer les constantes, on ne remplace pas une constante par une variable.
    \item \textbf{A refaire au moins une fois}
\end{itemize}

\subsection{Transformation de formule en clause}
Définition d'une clause : 
\begin{itemize}
    \item Pas de $ \exists  $ 
    \item Pas de $ \wedge  $ 
    \item Pas de $ \forall  $ implicite
\end{itemize}
\begin{enumerate}
    \item Mettre les quantificateurs au début : 
    \[
        F_3 : \forall x, \forall y (R(x,y) \to \exists z (R(x,z) \wedge R(z,y)))
    .\]
    Deviens 
    \[
        F_3 : \forall x, \forall y, \exists z (\neg R(x,y) R(x,z) \wedge R(z,y))
    .\]
    \item Skolenisation : supprimer les $ \exists  $ en inventant des constante. \begin{align*}
        F_1 &= \forall X, \exists Y, R(X,Y) \\
        F_2 &= \exists X, \forall Y, R(X,Y) \\
        F_3 &= \forall x, \forall y, \exists z (\neg R(x,y) R(x,z) \wedge R(z,y))
    \end{align*}
    Devient \begin{align*}
        F_1 &= \forall X, R(X, f(X)) \\
        F_2 &= \forall Y, R(x_0, Y) \\
        F_3 &= \begin{cases}
        \neg R(X,Y) \vee R(X, g(X,Y)) \\
        \neg R(X,Y) \vee R(g(X,Y), Y) \\
        \end{cases} 
    \end{align*}
\end{enumerate}

\section{Graph conceptuel}
\subsection{Représentation des connaissances}
\begin{itemize}
    \item "Rocher : \#" = "Le" rocher
    \item Bien choisir les relation dans les cercles
\end{itemize}
\subsection{Joiture et généralisation}
\begin{itemize}
    \item Jointure maximale : Est-ce que les deux phrases représente la même chose $\rightarrow$ Fusion ; /!\textbackslash au contradiction
    \item Généralisation : Généralisation de ce qu'on dit, vrais pour les deux. On vas au plus générale qui rend vrais les deux
    \item Subsumption : Un graph en subsume un autre si il est plus général
\end{itemize}

\section{Logique de description}
\subsection{$ \mathcal{FL}^- $ }
\begin{itemize}
    \item \textbf{S'entrainer pas compris} TD3 $\rightarrow$ Ca va en faite
    \item TBox : Concept atomique $ C \equiv D $, $ C \subseteq D \Leftrightarrow \forall x, C(x) \to D(x)$ 
    \item ABox : $ a : C, <a,b> : Role $ 
    \item Grammaire : pas de variable lol
    \item Bien utiliser les définition de $ \exists , \forall  $ \begin{align*}
        \exists R &= \{x \in \Delta | \exists y, (x,y) \in R\} \\
        \forall R.C &= \{x \in \Delta | \forall y, (x,y) \in R \to y \in C \}
    \end{align*}
\end{itemize}
\subsection{$ \mathcal{ALC} $ }
\begin{itemize}
    \item \textbf{S'entrainer RIEN RIEN compris} TD3 $\rightarrow$ Ca va en faite
    \item Same de $ \mathcal{FL}^- $ plus : 
    \item $ \exists R.C $ toujours role + concept atomique 
    \item $ \neg , \bot , \top  $ autorisé $\rightarrow$ Pratique
    \item Bien utiliser les définition de $ \exists , \forall  $ \begin{align*}
        \exists R.C &= \{x \in \Delta | \exists y, (x,y) \in R \wedge y \in C\} \\
        \forall R.C &= \{x \in \Delta | \forall y, (x,y) \in R \to y \in C\} \text{ (comme } \mathcal{FL}^- \text{)}
    \end{align*}
\end{itemize}

\subsection{Interprétation}
\begin{itemize}
    \item On a un graph avec des flèches au sens important.
    \item On regarde toujours les mondes de départ des flèches
    \item $ \exists s. \neg A $ se lit "Tous les mondes qui ont une flèche $ s $ qui pointe vers un monde qui vérifie $ \neg A $ "
\end{itemize}

\subsection{Méthode des tableaux}
TD4 mais pas beaucoup de correction
\begin{itemize}
    \item On veux prouver $ \phi  $  un truc vrais ou faux
    \item On part d'une TBox acyclique
    \item Puis notre première case du tableau contient $ Tbox \sqcap ABox \sqcap \phi  $ 
    \item Then on cherche à appliquer les bonnes règles pour arriver rapidement à notre objectif.
\end{itemize}

\section{Logique Modale}
\begin{itemize}
    \item On développe les formule $ \Box, \diamond  $ comme un arbre en explorant les possibilités.
    \item On peut donner des contres exemples.
    \item Règle de necessitation : $ M \models \phi \equiv M \models \Box \Phi $ 
    \item Penser que parfois les flèches de récursion ne sont pas dessiné
    \item /!\textbackslash . Au implication, parfois une traduction en vaut la penne $ a \to b $ toujours vrais pour les mondes où $ a $ est faux $\rightarrow$ vérifier surtout les mondes où $ a $ est vrais
    \item \textbf{s'entraîner vite fait fin exo 1 TD5}
    \item Penser au démo par l'absurde pour les trucs cons (TD5, fin exo3)
    \item Loi de Morgan \begin{itemize}
        \item $ \diamond \phi \equiv \neg \Box \neg \phi  $ 
        \item $ \Box \phi \equiv \neg \diamond \neg \phi  $ 
    \end{itemize}
\end{itemize}
Liste des axiomes logique épistémique S5 :
\begin{itemize}
    \item T : Réflexivité des mondes $ \forall w : (w,w) \in R $ : $ \Box \phi \to \phi  $ 
    \item D : Sérialité des mondes = aucun monde seul $ \forall w, \exists w' : (w, w') \in R $  : $ \Box \phi \to \diamond \phi  $ 
    \item 4 : Transitivité : classiquement en math : $ \forall x,y,z\in E\quad (x{\mathcal {R}}y\land y{\mathcal {R}}z)\Rightarrow x{\mathcal {R}}z. $ bah pareil avec les mondes: si je sais phi je sais que je sais phi : $ \Box \phi \to \Box \Box \phi  $ 
    \item 5 : Euclidienne : $ \diamond \phi  \to  \Box \diamond \phi  $ Ca implique qu'il existe un lien entre chaque monde presque : $ \forall w, w', w'', (w, w') \in R, (w',w'') \in R \to (w',w'') \in R $. D'après le prof c'est l'introspection negative : je sais ce que je ne sais pas.
    \item B : Symétrie des flèches : $ \phi \to \Box \diamond \phi  $ Implique qu'il existe toujours le chemin retour : $ \forall w, w', (w,w') \in R \to (w',w) \in R $ 
\end{itemize}

\subsection{Logique épistémique}
\begin{itemize}
    \item $ M, w_1 \models K_i p \equiv M, w_1 \models \Box p $ en utilisant les flèches indicées $ i $
    \item Croire possible $ p \equiv B_i \phi \equiv \diamond \phi \equiv \neg \Box \neg \phi \equiv \neg K \neg \phi  $ 
    \item Savoir si $ p $ : $ K^{si} p \equiv K p \vee K \neg p $ Bien le traduire lui, il est piège
    \item Savoir lequel parmi $ a,b,c \equiv K a \vee K v \vee K c $ 
\end{itemize}
\end{document}