\documentclass{article}
\usepackage[utf8]{inputenc}
\usepackage[a4paper, margin=2.5cm]{geometry}
\usepackage{graphicx}
\usepackage[french]{babel}

\usepackage[default,scale=0.95]{opensans}
\usepackage[T1]{fontenc}
\usepackage{amssymb} %math
\usepackage{amsmath}
\usepackage{amsthm}
\usepackage{systeme}

\usepackage{hyperref}
\hypersetup{
    colorlinks=true,
    linkcolor=blue,
    filecolor=magenta,      
    urlcolor=cyan,
    pdftitle={LRC},
    % pdfpagemode=FullScreen,
    }
\urlstyle{same} %\href{url}{Text}

\theoremstyle{plain}% default
\newtheorem{thm}{Théorème}[section]
\newtheorem{lem}[thm]{Lemme}
\newtheorem{prop}[thm]{Proposition}
\newtheorem*{cor}{Corollaire}
%\newtheorem*{KL}{Klein’s Lemma}

\theoremstyle{definition}
\newtheorem{defn}{Définition}[section]
\newtheorem{exmp}{Exemple}[section]
% \newtheorem{xca}[exmp]{Exercise}

\theoremstyle{remark}
\newtheorem*{rem}{Remarque}
\newtheorem*{note}{Note}
%\newtheorem{case}{Case}



\title{Logiques et représentations des connaissances}
\author{Charles Vin}
\date{S1-2022}

\begin{document}
\maketitle

\section{Introduction à la logique des propositions et des prédicats du premier ordre}
\begin{defn}[]
    Quelques définitions
    \begin{itemize}
        \item Expression : une description logique. Elle est vrais ou fausse
        \item Meaning : le sens qui relis l'expression vers sa référence
        \item Référence : représente le sens
    \end{itemize}

    Le langage des propositions : 
    \begin{itemize}
        \item Atome 
        \item Connecteur : \begin{itemize}
            \item Binary : $ \vee $ = "ou", $\wedge$ = "et" $\neg$ = négation 
            \item DIAPO
        \end{itemize}
    \end{itemize}

    Table de vérité : \begin{itemize}
        \item Un ligne = une "interprétation"
    \end{itemize}

    Une formule est \begin{itemize}
        \item Satisfiable : si vrais dans une interprétation
        \item Valide : si vrais dans toute les interprétation
        \item Unsatisfiable : faux sur toutes les interprétations
    \end{itemize}

    On peut avoir des fonctions qui représente une expression logique. \begin{itemize}
        \item Un terme représente une fonction qui renvoie un paramètre ou le tuple de paramètre d'une fonction.
        \item Un atome est une fonction qui renvoie un booléen.
    \end{itemize}
    En d'autre mot ça semble être en fonction de ce que la fonction renvoie un booléen ou un paramètre.
    
    Variable est : \begin{itemize}
        \item Liée (muette): Si elle est liée à un quantificateur, on peut donc changé librement son nom tant qu'on le change partout dans la formule
        \item Libre : C'est l'inverse, pas liée à un quantificateur donc libre dans la formule. Par exemple dans la formule $ \exists y, x<y $, la variable x est libre et y est liée. Cette formule représente le fait que “Il existe un nombre plus grand que x”. Au final ici x est en lien avec l'énoncé, il est fixe.
    \end{itemize}

    Diapo interpretation : rien compris

    A formula is "\textbf{valid}" if it is true in \textbf{all} the interpretation of \textbf{all} domains.
    \begin{exmp}[]
        \begin{itemize}
            \item Valide 
            \item DIAPO
        \end{itemize}
    \end{exmp}
    
    Un modèle est un couple $ \mathcal{M} = < \mathcal{D},i> $. Une valuation est une fonction $ v:\mathcal{V} \rightarrow  \mathcal{D} $. $ I_m(F) $ is the truth value. Y'a des propriétés de la truth value dans le DIAPO.

    A est une conséquence sémantique de B si A est vrais pour toutes les interpretations où B est vrais i.e. for all models if $ I_{m \mathcal{V}}(B) $ then $ I_{m \mathcal{V}}(A) $. Exemple $ A \rightarrow B $  est une conséquence sémantique. Je crois que le symbole c'est $ B \models A $ == A conséquence logique de B. 
\end{defn}

\begin{defn}[]
    Un système formel est composé d'un langage formel, un ensemble d'axiom et de règle d'inférence.

    \begin{exmp}[]
        Exemple dans le fiapo avec le système formel de Hilbert. 

        $ \vdash  $ représente une dérivation. On peut prouver A à partir de B.
    \end{exmp}
\end{defn}

\begin{defn}[Théorème]
    Any formula which is derived from the axoms by iteratively applying inference rules is a \textbf{theorem}.\\
    Notation : $ \vdash A $ means A is a theorem \\
    \begin{exmp}[]
        Exemple : $ \vdash (A \rightarrow A) $ 
        \begin{proof}[Preuve: ]
            Voir diapo c'est drôle
        \end{proof}
    \end{exmp}
\end{defn}

\begin{defn}[Démonstration]
    A \textbf{proof} of a theorem A is a finite sequence of formulae $ F_0, ..., F_n $ such that .... DIAPO mais osef un peu
\end{defn}

\begin{defn}[Symbolic system]
    \begin{itemize}
        \item Consistency : Each description of the symbolic system corresponds to an object in the reality i.e. DIAPO
        \item Completeness : each object of the reality can be described in the symbolic system $ \forall A \text{ if } \models A \text{ then } \vdash A$ 
    \end{itemize}
\end{defn}

\subsection{Automatic theorem proving}
\subsubsection{Tableau method}
\begin{enumerate}
    \item Normalisation : Transformation into \textbf{NNF - Negative Normal Form} : \textbf{The negations occurs only before atomic propositions}. i.e il faut développer les négations au max.
    \item Build a tableau : \begin{itemize}
        \item Root : The formula under NNF.
        \item Build successors of T using two rules $ R_{\wedge } $ et $ R_{\vee } $ VOIR LE DIAPO POUR LEUR DEFINITIONS
        \item On arrête lorsque l'on ne peux plus appliquer les règles.
    \end{itemize}
\end{enumerate}
\begin{defn}[]
    Un tableau peut être \begin{itemize}
        \item Contradictory 
        \item ... DIAPO 
    \end{itemize}
\end{defn}
\begin{exmp}[de la methode]
    DIAPO MDR OU YOUTUBE PLUTOT
\end{exmp}

L'avantage de cette méthode par rapport au tableau de vérité c'est qu'on a pas besoin de faire tous les cas possibles.

\paragraph*{Généralisation} Si je comprend bien on peut prendre des raccourcis avec des nouvelles règles. Pour la règles $ \alpha  $ on met dans le même tableau. Pour les règles $ \beta  $ on met dans deux tableaux différents. \textbf{Apprendre les tableau de règles}

\subsubsection{Resolution in propositional logic}
\begin{defn}[Une clause]
    \textbf{Ca tombe à l'exam d'après le prof}
    \begin{itemize}
        \item Un literal is either an atom or its negation
        \item Une clause is a disjunction of literals
    \end{itemize}
    \begin{rem}[]
        A clause is a logical entailment (implication) because $ (\neg A \vee B) $ is equivalent to $ (A \supset B) $ 
    \end{rem}
    \begin{exmp}[]
        $ even(X) \supset odd(successor(X)) $ 
    \end{exmp}
\end{defn}
\begin{thm}[]
    Any claused formula (sans variable libre) $ F $ can be transformed into a logically equivalent conjunction of clauses
    \begin{exmp}[]
        
    \end{exmp}
\end{thm}

\begin{defn}[Règle de résolution]
    S'applique uniquement sur les clauses.
    Pour prouver que $ S \models A $ il suffit de montrer que $ S \cup [neg A] $ est vide. Voir les exemples du DIAPO
\end{defn}

\subsubsection{Unification}
\begin{defn}[Substitution]
    A \textbf{substitution} is characterized by a infinite set of "

    On peut \textbf{composer} plusieurs substitution
\end{defn}
\begin{defn}[term instance]

\end{defn}

\begin{defn}[Pattern Matching]
    The term $ t_1 $ match with the term $ t_2 $ if and only if there exist a substitution $ \sigma  $ such that $ t_1 \sigma = t_2 $ 
\end{defn}
\begin{defn}[Unification]
    The terms $ t_1 $ and $ t_2 $  unify if and only if there exists a substitution $ \sigma  $ such that $ t_1 \sigma = t_2 \sigma  $ 
\end{defn}
\textbf{Pattern matching $\rightarrow$ Unification}

\subsubsection{Resolution in First Oder Logic}

\underline{Nouveau cours du 21/09} \\
\underline{Nouveau cours du 28/09} \\
Pas de diapo au moment du cours

\underline{Nouveau cours du 12/10} \\
\textbf{NOUVEAU PROF}
Modèle Lr : diamond et carré $\Leftrightarrow$ il existe et pour tout 

Modèle de Kripke = ensemble de monde (interprétation ?) . Puis on peut faire un graph sur comment relier les mondes. \begin{itemize}
    \item $ M, w_0\models p \rightarrow q $ c'est faux car dans le monde $ w_0 $ $ p $ est vrais mais $ q $ est faux
    \item $ M, w_0\models \Box p $ est vrais, on regarde si tous les mondes qui parte de $ w_0 $ on $ p $ de vrais.
    \item $ M, w_1 \models (p \vee r) \rightarrow \Box p$ ici on regarde dans $ w_1 $, la prémice de la fleche est vrais car p est vrais dans $ w_1 $, donc $ \Box p $ doit être vrais. On regarde dans $ w_3 $, $ p $ est faux donc l'implication est fausse. En revanche en partant de $ w_0 $ 
    \item $ M, w_2 \models \Box \Box p $ \begin{itemize}
        \item $w_2 \Box p$ \begin{itemize}
            \item $ w_2 p  $ Faux
            \item $ w_1 p  $ Vrais
        \end{itemize}
        \item $w_1 \Box p$ Faux
    \end{itemize}
    \item $ M, w_2 \models \Box \Box q $ Vrais
    \item $ M, w_2 \models \diamond (r \wedge \Box q) $ \begin{itemize}
        \item $ M, w_2 \models (r \wedge \Box Q) $ Vrais 
        \item $ M, w_1 \models (r \wedge \Box Q) $ Faux 
    \end{itemize} $\rightarrow$ VRAIS
\end{itemize}

Diapo 9 : \\
$ M, w \models \phi $ = La formule est statisfaire dans le monde $ w $ du model $ M $. \\
Si pas d'autre monde accessible ? alors $ M, w_1 \models \Box \phi  $ est vrais

Diapo 10 : \\
Loupé bruh

Diapo 12 : \\
Démonstration par l'absurde 

Système normal K (diapo 13): on veut avoir un système de preuve simple, comme celui de Hilbert. On a un axiome sous la forme d'une implication comme pour Hilbert. Mais on simplifie un peu par rapport à Hilbert niveau démonstration, je crois pas l'axiome $ K $ . \\
Déduction (diapo 14) : On a prouvé $ p \rightarrow p \vdash \Box p \rightarrow \Box q $ (dans un univers ou tous les mondes vérifie $ p \rightarrow q $ alors $ \Box p \rightarrow \Box q $). Attention ça n'est pas équivalent a $ \vdash p \rightarrow p \vdash \Box p \rightarrow \Box q$, dans celle là on ne vérifie pas l'HP que tous les mondes on $ p \rightarrow q $. 

Correction et complétude : 
\begin{itemize}
    \item $ \vdash  $ == démonstration dans un monde
    \item $ \models  $ == démonstration dans tous les mondes
    \item Correction : 
    \item Complétude :
    \item $\rightarrow$ Le système $ k $ est correct et complet vis à vis de l'ensemble des modèle de Kripke. 
\end{itemize}

CCL : 
\begin{itemize}
    \item Une formule est satisfiable DIAPO
    \item Non validité : on montre un contre exemple
    \item Insatisfiabilité : 
    \item Validité : \begin{itemize}
        \item par raisonnement sémantique
        \item Avec la méthode des tableau et $ \lnot \phi  $ UNSAT
        \item Un dernier truc pas compris
    \end{itemize}
\end{itemize}

\begin{exmp}[]
    Montrer que $ \Box \phi \rightarrow \Box \Box \phi  $ n'est pas valide. \\
    $\Rightarrow$ Contre exemple : voir diapo 18
\end{exmp}

Théorie de la correspondance : 
\begin{itemize}
    \item C'est une série d'axiome 
    \item Preuve de la réflexivité dans le DIAPO
    \item Sérialité : globalement on kick out les monde sans fleche 
    \item Transitivité : "Si quelque chose est vrais à un pas de distance, alors il est vrais à deux pas de distance également"
    \item Euclidienne : "Si je crois possible que phi, je sais que je crois possible de phi"
    \item Symétrie : 
\end{itemize}

\end{document}